\chapter*{Colophon}\addcontentsline{toc}{chapter}{Colophon}

These notes are set in Latin~Modern,
an updated version of Donald Knuth's Computer~Modern typeface.
The typeface is a Didone design, meaning a high contrast between thin and thick lines.
Knuth designed it after mathematics textbooks set in metal type.
I am personally not that fond of Computer Modern,
especially when reading on computer screen.
I still kept it so that the notes are honest to \LaTeX's default behaviour.

As you can see from the source code,
there is nothing too special in these notes.
The highlighted \emph{Gotcha!} blocks are created with \pkg[production notes]{thmtools}.
There are also a few custom commands for printing command and package names:
these commands also contribute to the index.

The largest programming trick was the examples.
Code for the examples appears in the \verb|.tex| code inside a \env{VerbatimOut} environment.
The code is stored into a temporary file, and then read twice:
once to show the code, and a second time to evaluate the output.

Due to this design, the examples are guaranteed to really work.
However, a few examples include a few lines of secret extra code.
This code mainly hides the fact that the examples live inside a minipage,
and restores some customizations to \LaTeX{} defaults.

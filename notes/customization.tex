\chapter{Further customization}

\section{Alternative \LaTeX{} compilers}

During the years, the preferred output format of \TeX{} has changed:
\begin{enumerate}
\item The original implementation of \TeX{} produced DVI~files
    -- this was a device-independent format that could be processed into commands
    to be sent to the printer or photo\-type\-setting machine.
\item Simultaneously in the 1980s, Adobe developed the PostScript format.
    PostScript is a language for expressing complex drawing commands to printers.%
\footnote{Processing drawing commands requires processing power and memory.
The original Apple laser printers were \emph{more powerful} than the
Macintosh computers used to lay out the documents!}
    It became the universal standard,
    and in the 1990s a lot of \TeX{} documents were put online as \verb|.ps| files.
    These files were produced from DVI~files by a converter.
\item PDF (Portable Document Format), also by Adobe,
    was created in the 1990s and has become the dominant standard.
    PDF encapsulates a subset of PostScript, image data, and metadata about the document.
\end{enumerate}

Originally, PDF~files were produced by first converting DVI to PostScript and then to PDF.
In early 2000s, a new implementation of \TeX{} appeared: \prg{pdftex}.
It produces PDF~files directly and supports some special features of the format:
For example, the \pkg{pdflscape} package can tell the PDF reader
to display a landscape page in rotated mode.

\begin{technote}
The only difference between \textbf{pdftex} and \textbf{pdflatex}
is whether the \LaTeX{} macros are automatically loaded.
These notes make no effort at being consistent with the names.
\end{technote}


Another development was with the character sets.
Original \TeX{} used 7~bits per character, because 128~characters is enough for anybody
(that is, anybody writing in English).
In the 1980s, this was recognized as unsustainable,
so the character set was \emph{doubled} into 8~bits.
Any further extensions to that are based on swapping fonts on the fly.

This limitation applies also to \prg{pdftex},
but in the current\footnote{Read: past 20~years.} age of Unicode there are now alternatives.
The two major alternative compilers are
\begin{itemize}
\item \prg{luatex}, which is based on \prg{pdftex} source code.
    It uses Unicode as its native character set.
    As the name suggests, extension packages can be also written in the Lua programming language.
    Some packages like TikZ use the Lua support for a speed boost.
\item \prg{xetex} is an independent Unicode-native development
    that uses a completely different PDF backend.
    It produces a bit smaller PDF files than \prg{luatex},
    but it is less actively maintained.
\end{itemize}

Both of these ``modern'' engines support loading arbitrary OpenType fonts
with the \pkg{fontspec} package, unlike \prg{pdftex}.
(This is discussed in \Cref{sec:typefaces}.)
The native use of Unicode also means that letters like `ä'
really are output as the glyph `ä',
not \texttt{\textasciidieresis a} superimposed on top of each other
-- something you only see when copying text from a PDF~file.

\begin{overleaf}
You can choose the compiler in the settings menu that can be opened at the top-left corner.
Most editor programs have a similar menu.
\end{overleaf}

\begin{practices}
Based on the discussion above,
it would be natural to suggest always using \prg{luatex}.
For personal writing, either of the Unicode engines should be the first choice.

Alas, arXiv and many publishers still use \prg{pdftex} in their pipeline.
This means that we are still stuck with making our articles \prg{pdftex}-compatible.
Hopefully the situation will change soon enough.
\end{practices}



%
%
%
\section{Accessibility of documents}

The PDF file format is principally a set of instructions to a printer:
draw a curve here, another there, and so on.
This poses a problem for accessibility.
Some examples of accessibility issues are:
%
\begin{itemize}
\item Can a ``screen reader'' program read the document aloud for a blind person?
\item Can a visually impaired person increase the color contrast on graphics?
\item Can the text from the document be copy-pasted into another program?
\end{itemize}
%
Note that the last point might not sound like an accessibility issue,
but it is caused and solvable by the same reasons.

There are three layers of accessibility work going on:
%
\begin{enumerate}
\item The PDF format can include semantic metadata about the document contents.
    \LaTeX{} is increasingly better at writing this metadata,
    but it might require you to opt in to the new behavior.
\item \LaTeX{} can be compiled into another format, such as HTML web page.
    This approach is increasingly used by major publishers and arXiv.
    It requires you to use a sensible subset of \LaTeX.
\item Source code is very accessible to a screen reader program;
    a well-written \verb|.tex| file is a good workaround.
    This still requires you to write clean code.
\end{enumerate}
%
We will briefly explore each of these layers below.

\begin{practices}
Accessibility is for everybody.
Most people benefit from some accommodations at some point in their life.
Even without an ``obvious'' disability, think about reading your document
\begin{itemize}
    \item on an ebook reader with grayscale screen (visual impairment),
    \item on a mobile phone in bright sunlight (ditto), or
    \item by a sleep-deprived parent between meetings (cognitive impairment).
\end{itemize}
Not all accessibility issues have technological solutions
-- they also require you to think about how you present your message.
A complicated figure is hard to read, no matter the technology used to produce it.
\end{practices}


%
%
\subsection{PDF accessibility}

This is the core accessibility work done by the \LaTeX{} team during the last few years.
The goal is to embed semantic metadata about the document into the PDF format.
A lot of the semantics is in the \verb|.tex| source in form of sectioning commands and such.

The translation is hampered by the need to rewrite \LaTeX{} internals without breaking compatibility,
and all the abuse of commands to get semantically incorrect but visually good results.

To enable PDF metadata, you should do the following steps:
\begin{itemize}
\item Use a modern \LaTeX{} compiler like \prg{luatex} or \prg{xetex}.
    Since \prg{luatex} is more actively maintained, it is preferable.
\item Add \cmd{DocumentMetadata} as the first line (before \verb|\documentclass|) of the code.
\item Load the \pkg{hyperref} package.
\item Consider loading the \pkg{unicode-math} package
    (and remove other mathematics symbol and font packages;
    see page~\pageref{rem:math unicode}).
\end{itemize}

\begin{warning}
Since some behavior changes subtly,
you should take these into use only in new projects
and only when everybody involved (including the publisher) uses a Unicode-native \LaTeX{} pipeline.

Be extra careful with checking the output when converting old projects,
or if you need to compile the project with both old and new toolsets.
\end{warning}

\begin{warning}
As of April~2024, arXiv \emph{does not} use Unicode-native \LaTeX{} tools;
their process involves \prg{pdftex}.
For arXiv submissions, follow the next section instead.
\end{warning}

The \cmd{DocumentMetadata} command tells \LaTeX{} that you opt in
to new and rewritten behaviors.
For example, it enables some new features in the \pkg[opt-in features]{hyperref} package.

\begin{latexthree}
This work is still heavily in progress.
There are some arguments to \cmd{DocumentMetadata} that enable further in-development features.
As of April~2024 these include tagging of mathematics and tables.

Since these arguments are unstable, they are not documented here
and should not be used in ``real'' documents quite yet.
You can follow the progress through \LaTeX{} release newsletters:
\begin{quote}
\url{https://www.latex-project.org/news/}
\end{quote}
\end{latexthree}

In order for the tagging to work,
you need to produce semantically correct code.
That is, use \verb|\subsection*| when you need an unnumbered subsection,
and only then -- do not abuse it for bold, heavy text.
Similarly, use \verb|\emph| to emphasize,
instead of \verb|\textit| or \verb|\textbf| that carry only visual meaning.


%
%
\subsection{HTML conversion}

Another option is to skip PDF altogether.
The most common alternative format is HTML, the technology behind web pages.
It has a few benefits:
%
\begin{itemize}
\item It is processed on the user device, so e.g.\ colours can be customized by the reader.
\item The text can be reflowed based on the screen size and magnification factor.
\item Existing screen reader programs can already read out web pages.
\end{itemize}
%
However, there are a few drawbacks too:
%
\begin{itemize}
\item \TeX{} is built on the model of printing on paper; it assumes a fixed layout in many places.
\item The usual \LaTeX{} compilers only produce PDF output or equivalent.
\end{itemize}

The HTML approach is now pursued by several publishers and arXiv,
together with e.g.\ American Mathematical Society.
ArXiv uses a tool called \mbox{LaTeXML}%
\footnote{\url{https://github.com/brucemiller/LaTeXML}}
that is essentially a new \TeX{} compiler that selectively reimplements some packages.

Since \LaTeX{} is already semantically focused, this works surprisingly well.
A lot of packages are already supported.
Importantly, TikZ can output its pictures also in the SVG (Scalable Vector Graphics) format
used together with HTML.\index{TikZ!accessibility}

\begin{practices}
Load and use only the packages that you really need.
Prefer mainstream packages supported by the arXiv converter.
ArXiv has a help document outlining some more best practices for HTML conversion.\footnotemark
\end{practices}
\footnotetext{\url{https://info.arxiv.org/help/submit_latex_best_practices.html}}



%
%
\subsection{Accessible source code}

Both of the above methods require you to use semantically meaningful commands:
otherwise, the automated tools are unable to do tagging/conversion.
Yet writing good semantic \LaTeX{} code is also an accessibility feature:
you can always read the source code.

\begin{remark}
Source code posted to arXiv is public and downloadable.
\end{remark}

If you want to set vectors in bold font, then do not write \verb|\mathbf| but
redefine \verb|\vec| to set vectors in bold style:
%
\begin{VerbatimOut}{\jobname.tmp}
% Preamble
\renewcommand{\vec}{\mathbf}

$\vec a$
\end{VerbatimOut}
\ShowExample
%
Now it is immediate from the source code that \verb|a| refers to a vector.
It can not be confused with other things marked with boldface.
Besides, if you change your mind about the style,
you only need to change one command in the preamble.


%
%
%
\section{Typefaces}\label{sec:typefaces}

We will discuss two methods for changing the typeface in your document.
In neither case, we do not comment too much on the aesthetic qualities
-- use at your own risk!

\begin{warning}
When changing fonts, especially the one used for mathematics,
make sure that all the special characters still appear correctly.
The character support of different fonts varies wildly.
\end{warning}


%
%
\subsection{Fonts via packages}

Several common typefaces are packaged on CTAN as ready-to-use packages.
This is a preferable way since the package author might have done some necessary
typographical tweaks already for you.

To find packaged typefaces, you can use the \emph{\LaTeX{} Font Catalogue}%
\footnote{\url{https://tug.org/FontCatalogue/}}
maintained by \TeX{} Users' Group.
Each catalogue entry displays the code needed to load the font with typical installations.

In the \emph{Font implementation} section,
you should check what types are available.
Type~1 fonts can be used with pdfTeX,
whereas LuaTeX and XeTeX additionally support (and prefer!) OpenType or TrueType fonts.

Let us see a few examples of common fonts.
First, for comparison, here is a small standard text written with the default Latin Modern typeface
(the $\lesssim$ symbol comes from \textsf{amssymb}):
%
\EmbedPdfPage{examples/font_default.pdf}{1}

The Palatino typeface designed by legendary Hermann Zapf is a classic
and commonly used also in the \TeX{} world.
It is packaged as the PX text and math fonts.
The font automatically comes with \verb|\lesssim|, so we do not need to load the symbol package.
(See the next subsection for a different version with OpenType fonts.)
%
\begin{ExampleCode}
\usepackage{newpxtext,newpxmath}
% No need to load amssymb
\end{ExampleCode}
%
\EmbedPdfPage{examples/font_px.pdf}{1}

\begin{technote}
Typeface designs cannot be copyrighted, but the names are trademarks.
This is why near-identical implementations of the same typeface can appear with very different names.
\end{technote}

As a slightly different example, we use the Noto font.
This typeface design by Google aims to contain \emph{all Unicode characters}.%
\footnote{Its name comes from \textbf{no to}fu, since the rectangular `missing character' symbol
in web browsers is sometimes called `tofu'.}
It too can also be loaded with the mechanism described in the next section.
%
\begin{ExampleCode}
\usepackage{notomath}
% No need to load amssymb
\end{ExampleCode}
%
\EmbedPdfPage{examples/font_noto.pdf}{1}

For one more example, let us see a sans serif typeface.
The Source~Sans typeface is an open source font by Adobe.
When the package is loaded with the \verb|default| option,
sans serif becomes the default font family.
However, it does not come with mathematics support.
Without further tweaks, the display mathematics comes in a very different typeface!
%
\begin{ExampleCode}
\usepackage[default]{sourcesanspro}
\usepackage{amssymb}
\end{ExampleCode}
%
\EmbedPdfPage{examples/font_sourcesans.pdf}{1}

Many sans serif packages have options to make sans serif the default style.
The same can be achieved manually with
\begin{ExampleCode}
\renewcommand{\familydefault}{\sfdefault}
\end{ExampleCode}


%
%
\subsection{OpenType fonts via fontspec}

\begin{remark}
This section only applies to LuaTeX and XeTeX.
\end{remark}

Like character sets, also font formats have undergone a parallel evolution
between \TeX{} and the rest of the world.
Nowadays, fonts are distributed in the OpenType format (and its predecessor TrueType).

The \pkg{fontspec} package allows loading OpenType/TrueType fonts
on the Unicode-native compilers.
In the preamble, the default roman, sans serif, and typewriter fonts are
set with the respective commands:%
\indexcmd{setmainfont}\indexcmd{setsansfont}\indexcmd{setmonofont}
%
\begin{ExampleCode}
% Note: only works on Windows, where these are installed!
\setmainfont{Times New Roman}
\setsansfont{Comic Sans MS}
\setmonofont{Consolas}
\end{ExampleCode}
%
In this example, we used the system font names.

\begin{technote}
A XeTeX-specific issue:
If \pkg{fontspec} cannot find the font,
XeTeX tries to fall back to the old MetaFont technology.
This manifests itself as log messages about \prg{mktexmf} failing to build (non-existent) files.
\end{technote}

It is also possible to give file names to fonts,
if you have the font files in the same folder tree as your document.%
\footnote{Or anywhere on \TeX's search path.}
In this case, you might need to specify file names for each weight separately:%
\footnote{This example takes advantage of it,
and uses the Book and Semibold weights instead of Regular and Bold.}
%
\begin{ExampleCode}
\setmainfont{FiraSans-Book.otf}[
    BoldFont={FiraSans-SemiBold.otf},
    ItalicFont={FiraSans-BookItalic.otf}]
\end{ExampleCode}

Instead of overriding the default fonts,
it is also possible to define new font families with \cmd{newfontfamily}:
%
\begin{ExampleCode}
% Preamble
\newfontfamily{\comicfont}{Comic Sans MS}

% Document
Some people hate {\comicfont Comic Sans},
but its history is often {\comicfont misunderstood}.
\end{ExampleCode}
%
All these font loading command support \emph{a lot of} optional arguments
-- see the package documentation.
These can be used to control the appearance of the typeface
and to enable stylistical variants of some letters.

For example, when using typefaces not originally designed to play together,
it might be useful to slightly adjust the sizes of fonts:
%
\begin{ExampleCode}
% Fits a bit better with Latin Modern
\newfontfamily{\comicfont}{Comic Sans MS}[Scale=0.9]
\end{ExampleCode}

The \pkg{unicode-math} package introduces the further \cmd{setmathfont} command
that works just like the commands above, only for the math mode.
The font needs to include the mathematical symbols,
which most fonts do not have!
This package also turns all mathematical symbols into proper Unicode characters
that can be copied from and searched within the PDF.

Let us illustrate one nice combination of OpenType text and mathematics fonts.
\TeX{} Gyre Pagella is another version of Palatino, usually included with \LaTeX{} distributions.
The Asana~Math font is also part of \TeX~Live, but here we load it from an included file:
%
\begin{ExampleCode}
\usepackage{fontspec}
\setmainfont{TeX Gyre Pagella}
\usepackage{unicode-math}
\setmathfont{Asana-Math.otf}
% No amssymb loaded
\end{ExampleCode}
%
\EmbedPdfPage{examples/font_pagella.pdf}{1}


\begin{technote}
Another odd corner of copyright law:
typeface designs cannot be copyrighted, but \emph{font files} are!
The reason is that the OpenType format is a small programming language,
which makes fonts be classified as computer programs.

Only use fonts that you have the right to use:
those that come with your computer system,
those you have bought,
or ones available under a permissive license (like the SIL~Open Font License).
As long as you have the license to use a font,
you can freely distribute any documents typeset with it.
\end{technote}



%
%
\subsection{Point sizes and microtypography}

Let us just briefly comment on two more nuances of setting text.
\LaTeX{} provides the semantically meaningful font size commands (page~\pageref{sec:font size}),
but what if you need to specify the exact point size?

The low-level \cmd{fontsize} command can be used in this case.
It takes two arguments: the desired size of the font, and the new baseline skip value.
If the font does not support arbitrary size, the closest match is chosen.
This declaration must be followed by \cmd{selectfont},
which does the hard work of loading the font.
%
\begin{VerbatimOut}{\jobname.tmp}
\fontsize{12pt}{14pt}\selectfont
This is set in 12~pt font,
with 14~pt baseline skip.
Some extra text to illustrate
the line heights.
\end{VerbatimOut}
\ShowExample

\begin{technote}
\LaTeX{} might need to do a fair bit of work to prepare a font for the given size,
so you might hit some internal limitations if there are too many different sizes in use.
\end{technote}

Some more typographical tweaks are provided by the \pkg{microtype} package.
Once loaded, it adds features like hanging punctuation
(punctuation characters extending into the margins) and some tiny spacing adjustments.
We are now talking of extremely small typographical details.

\begin{warning}
Support for \pkg{microtype} features depends a lot on the compiler.
If you use the package, be aware that the page layout can change significantly
between different compilers.
\end{warning}

\begin{warning}
Some fonts might not pair well with the built-in adjustment parameters.
\end{warning}


%
%
\subsection{Emoji}\label{sec:emoji}

Emoji are Unicode characters, so in principle they can be included just as is in the source code.
The problem is choosing a font that can display the characters.

The \pkg{emoji} package automatically chooses an available emoji font
(that is, the output will depend on your operating system!).
It only works with recent LuaTeX (2020 or newer).
The command provides readable shorthand names for the characters (listed in documentation):
%
\begin{ExampleCode}
% Preamble
\usepackage{emoji}

% In text
\emoji{thinking-face}
\end{ExampleCode}

There is also the manual way of loading the emoji font with \pkg[emoji]{fontspec}:
%
\begin{ExampleCode}
% Preamble
\usepackage{fontspec}
\newfontfamily{\emojifont}{Noto Color Emoji}[Renderer=Harfbuzz]

% In text
{\emojifont ...}
\end{ExampleCode}
%
Here, \verb|...| should be replaced with the emoji
(not displayed here, since these notes do not have an emoji font set up!).
You need to install the emoji font or replace it by the system emoji font.
(On Windows, it is called \texttt{Segoe UI Emoji};
on macOS it is \texttt{Apple Color Emoji}.)

\begin{overleaf}
The Overleaf editor cannot handle the Unicode emoji characters as of May~2024.
If you use Overleaf, you need to use the \pkg{emoji} package.
\end{overleaf}

\begin{warning}
With the manual way, black-and-white emoji can be displayed by both XeTeX and LuaTeX,
but color emoji requires LuaTeX with the HarfBuzz renderer.
If you use \pkg[emoji]{fontspec} to load an emoji font,
you need to pass \verb|Renderer=Harfbuzz| as an optional font feature.
This option is effective only with LuaTeX.
\end{warning}

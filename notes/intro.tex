\addtocounter{chapter}{-1}
\chapter{Introduction}


These notes were created for a course called \emph{Advanced \LaTeX},
offered to PhD students at University of Helsinki in spring~2024.%
\footnote{I dislike the name of the course.
As you will find out, we do not talk very much about advanced topics like
\TeX{} internals or package development.
A better name would be \emph{Professional \LaTeX} or \emph{A second course in \LaTeX},
as these notes are named.}
They attempt to give a somewhat unified view of what \LaTeX{} is,
how it works, and how it should be used in producing scientific literature.
They are much larger than can be covered in a two-week intensive course
-- my hope is that they can work as a reference as well.

These notes owe a lot to \emph{The \LaTeX{} Companion} \cite{TLC}
by Frank Mittelbach and other \LaTeX{} team members.
The third edition of the book is a treasure trove for a serious \LaTeX nician,
but at three kilograms (printed version) it is not an investment for everybody.
I have tried to distill some of the wisdom%
\footnote{While undoubtedly introducing some un-wisdom of my own.}
into these notes (which are still not what you'd call lightweight).

Of course, the authoritative reference for each package is the documentation hosted on
the Comprehensive \TeX{} Archive Network (\url{ctan.org}).



%
%
\section{What you need}

I assume that you already know the basics of \LaTeX{}
-- ideally, you have written a Bachelor's or Master's thesis or an article with it.

You need to have an up-to-date \LaTeX{} environment.
If you use a local installation, pause now for a moment
and check whether the packages are up to date.
If you use Overleaf, then the system is already good to go.

I encourage you to compare different editor programs.
As programmers can testify, having a good editor makes coding much more pleasant.
It is good to get familiar with the keyboard shortcuts your editor offers.

Since there are so many editors and they change much faster than base \LaTeX,
these notes do not discuss their use.

\begin{overleaf}
As an exception, there are a few blocks like this that discuss Overleaf.
I felt that Overleaf is worth mentioning, since so many people use it nowadays,
and it has some differences to locally installed tools.
\end{overleaf}


%
%
\section{About this version}

This is Version~1.1 of the notes, published at the end of the Spring~2025 course.

This version contains some fixes to the 1.0~version,
but it is still in need of a good copy editor.
All red \textcolor{red}{\textbf{TODO}} notes that were there at the end of the~2024 course
have just been turned into blue \textcolor{blue}{\textbf{FUTURE}} notes,
only visible in the working draft available on GitHub (see below).

\future{Particular examples that need work are the index,
and probably giving a still more consistent overview of the design principles of (La)\TeX.}

Maybe someday I will have a chance to properly polish these notes.
Any comments and suggestions are more than welcome.



%
%
\section{See how it is made}

\noindent{\Huge\faCreativeCommons\faCreativeCommonsBy}
These notes are licensed under the Creative Commons Attribution~4.0 license.%
\footnote{\url{creativecommons.org/licenses/by/4.0/}}
You are free to copy and redistribute them as you like.
You can also modify and reuse them freely,
under the condition that you indicate clearly the original author
and whether you have modified the work.
You can see the license terms for complete description.

\bigskip\noindent{\huge\faGithub}
The source code and latest PDF~release for these notes are available on GitHub:
\url{github.com/polsys/2nd-course-latex}.

You are invited to see how I built these notes
-- I tried to follow my own guidelines, but of course you can disagree with some choices.
In the end, there is no one true path to \TeX nical enlightenment.

If you spot a typo or an error, you can send a pull request with the correction,
file a GitHub issue, or send me an email about it.
You can find my up-to-date contact information at \url{petri.laarne.fi}.

\bigskip\noindent
I hope that you find these notes useful!

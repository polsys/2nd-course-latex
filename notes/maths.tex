\chapter{Mathematics layout}\label{sec:mathematics}

Plain \TeX{} already provides a lot of facilities for typesetting mathematics,
and this is further extended by \LaTeX{} and the \pkg{amsmath} packages.
The \pkg{amsmath} package is essentially as old as \LaTeX{}.
It is automatically loaded by the Americal Mathematical Society document classes like \verb|amsart|,
so you might not even need to load it.

The package works well but has not been significantly updated since the nineties.
Some additions and bug fixes are collected in the \pkg{mathtools} package.
If you use a non-AMS document class, you can load \pkg{mathtools} instead of \pkg{amsmath}
to get slightly improved typesetting.

\begin{technote}
The \pkg{amsmath} package used to be tied to the AMS document classes,
but it is nowadays maintained by the \LaTeX{} core team.
It is one of the packages guaranteed to be present on every \LaTeX{} distribution.
\end{technote}


%
%
%
\section{Equation environments}

Standard \LaTeX{} provides the \verb|\[ ... \]| syntax for
creating a mathematics display (as opposed to inline mathematics with \verb|$ ... $|).
This environment does not support equation numbering or line breaks;
for those, \pkg{amsmath} provides a lot of options.

\begin{warning}
Plain \TeX{} used the \verb|$$ ... $$| syntax for display mathematics.
Do not use it -- the \LaTeX-style environment has some hooks and accessibility features
that the \TeX{} syntax does not have.
\end{warning}


%
%
\subsection{Single numbered equation}

The basic equation environment of \pkg{amsmath} is \env{equation}.
It sets its contents on a single line and numbers the equation:
%
\begin{VerbatimOut}{\jobname.tmp}
\begin{equation}
i \partial_t u - \Delta u = -u^3.
\end{equation}
\end{VerbatimOut}
\ShowExample

If you'd rather have an unnumbered equation,
you can use the \verb|equation*| environment.
It is essentially equivalent to \verb|\[ \]| of \LaTeX.

To customize the equation number, you can use the \cmd{tag} command.
%
\begin{VerbatimOut}{\jobname.tmp}
\begin{equation}
i \partial_t u - \Delta u = -u^3.
\tag{NLS}
\end{equation}
\end{VerbatimOut}
\ShowExample
%
The tag also appears in cross-references to this equation.
The tag is read in text mode, so any mathematical symbols need to be wrapped with \verb|$|.
For example, you would write \verb|\tag{$\ast$}| to produce ($\ast$).

If you want to change whether equations are numbered globally or per section,
use the \cmd{counterwithin} mechanism described in \Cref{sec:counters}:%
\footnote{The \pkg{amsmath} package also defines \cmd{numberwithin},
which is functionally identical to \cmd{counterwithin}.
It does not matter which one you use.}
%
\begin{ExampleCode}
\counterwithin{equation}{section}
\end{ExampleCode}


%
%
\subsection{Single equation on many lines}\label{sec:math split}

You can break a long expression into multiple lines with the \env{multline} environment.
Only one equation number is produced (none if you use \verb|multline*|).
Line breaks are specified with \verb|\\|:
%
\begin{VerbatimOut}{\jobname.tmp}
\begin{multline}
a^5 + 5 a^4 b \\
+ 10 a^3 b^2 + 10 a^2 b^3\\
+ 5 a b^4 + b^5
\end{multline}
\end{VerbatimOut}
\ShowExample
%
The first line is left-aligned, the last line is right-aligned,
and the rest are centered.

If you need to control the alignment,
you can use the \env{split} environment.
This environment needs to be put inside \verb|equation| (or \verb|equation*|).
The alignment point is denoted by \verb|&|:
%
\begin{VerbatimOut}{\jobname.tmp}
\begin{equation}
\begin{split}
(a+b)^2
&= (a+b)(a+b)\\
&= a^2 + 2ab + b^2.
\end{split}
\end{equation}
\end{VerbatimOut}
\ShowExample

Sometimes, you need to align with things that are not really there.
Let us produce a slightly different version of the previous example:
%
\begin{VerbatimOut}{\jobname.tmp}
\begin{equation}
\begin{split}
&\mathrel{\phantom{=}} (a+b)^2\\
&= (a+b)(a+b)\\
&= a^2 + 2ab + b^2.
\end{split}
\end{equation}
\end{VerbatimOut}
\ShowExample
%
The \cmd{phantom} command reserves enough space on the first line for $=$,
even though it is not printed there;
the \cmd{mathrel} command ensures that the spacing is that surrounding $=$ as well.
This ensures that the expressions are aligned.
The same effect can not be attained with moving the alignment symbol,
since the \TeX{} spacing algorithm (described below) cannot see across the \verb|&|:
%
\begin{VerbatimOut}{\jobname.tmp}
Bad spacing after $=$ sign:
\begin{equation}
\begin{split}
&(a+b)^2\\
= &(a+b)(a+b)\\
= &a^2 + 2ab + b^2.
\end{split}
\end{equation}
\end{VerbatimOut}
\ShowExample



%
%
\subsection{Many equations, many lines}

The \env{gather} environment collects equations, each centered.
%
\begin{VerbatimOut}{\jobname.tmp}
\begin{gather}
(a+b)^2 = a^2 + 2ab + b^2,\\
a^2 - b^2 = (a+b)(a-b).
\end{gather}
\end{VerbatimOut}
\ShowExample

This environment also supports \env{split} on one or more sub-equations:%
\footnote{Now we wrap the phantom equals sign in \cmd{mathord},
which makes the $+$ sign act as binary plus, not unary plus.}
%
\begin{VerbatimOut}{\jobname.tmp}
\begin{gather}
(a+b)^2 = a^2 + 2ab + b^2,\\
\begin{split}
(a+b)^3 &= a^3\\
&\mathord{\phantom{=}} + 3a^2 b + 3ab^2 + b^3.
\end{split}
\end{gather}
\end{VerbatimOut}
\ShowExampleBelow

You can customize the tag of each individual equation with \cmd{tag},
and also suppress an individual tag with \cmd{notag}.
These commands can appear anywhere before the \verb|\\| that ends the particular equation.
Again, the \verb|gather*| environment suppresses all tags
except those explicitly created with \cmd{tag}.
%
\begin{VerbatimOut}{\jobname.tmp}
\begin{gather}
(a+b)^2 = a^2 + 2ab + b^2,\\
(a+b)^3 = a^3 + 3a^2 b + 3a b^2 + b^3, \notag\\
a^2 - b^2 = (a+b)(a-b). \tag{$\ast$}
\end{gather}
\end{VerbatimOut}
\ShowExampleBelow

If you need to have several aligned equations, you can use the \env{align} environment.
In comparison to \env{split}, each line is now numbered individually:
%
\begin{VerbatimOut}{\jobname.tmp}
\begin{align}
(a+b)^2
&= (a+b)(a+b)\\
&= a^2 + 2ab + b^2.
\end{align}
\end{VerbatimOut}
\ShowExample
The same remarks about \cmd{tag} and \cmd{notag} with \verb|gather| apply here.
The \env{align} environment also supports more than one alignment point.


\begin{warning}
There is also an \obsenv{eqnarray} environment provided by base \LaTeX.
It is far less sophisticated and generally uglier than
anything you can achieve with the environments presented here,
so I would avoid it.
\end{warning}

If you want to number the equations as subequations,
it can be done by wrapping the environment inside \env{subequations}:
%
\begin{VerbatimOut}{\jobname.tmp}
\begin{subequations}  
  \begin{align}
    (a+b)^2
    &= (a+b)(a+b)\\
    &= a^2 + 2ab + b^2.
  \end{align}
\end{subequations}
\end{VerbatimOut}
\ShowExample


If your equation environment is very long,
it might be beneficial to allow page breaks.
The display environments presented here do not allow page breaks by default.
By putting a \cmd{displaybreak} command before the \verb|\\|,
you indicate that a page break \emph{is allowed}.

The \cmd{allowdisplaybreaks} command permits page breaks in displays everywhere in its scope.
If you put it inside an environment, it applies to that environment only;
if you put it in the preamble, it applies to all environments.


%
%
\subsection{Cases}

To group several expressions with vertical brackets,
you can use the \env{cases} environment inside an equation environment.
This environment supports a single alignment \verb|&|:
%
\begin{VerbatimOut}{\jobname.tmp}
\begin{equation}
f(n) = \begin{cases}
    n^2, & n > 0,\\
    -n, & n \leq 0.
\end{cases}
\end{equation}
\end{VerbatimOut}
\ShowExample
%
If you mostly have text following the alignment character,
you can use the starred environment.
It interprets the right-hand side of \verb|&| in text mode:
%
\begin{VerbatimOut}{\jobname.tmp}
\begin{equation}
f(n) = \begin{cases*}
    n^2+1, & if $n$ is even,\\
    n+7, & if $n$ is odd.
\end{cases*}
\end{equation}
\end{VerbatimOut}
\ShowExample

They also have a right-aligned cousin \env{rcases}:
%
\begin{VerbatimOut}{\jobname.tmp}
\begin{equation}
\begin{rcases}
i \partial_t u - \Delta u = -u^3,\\
\partial_{tt} u - \Delta u = -u^3
\end{rcases}
\text{ some dispersive PDE}
\end{equation}
\end{VerbatimOut}
\ShowExampleBelow

If you want to number lines in the \verb|cases| environment individually,
check out the \pkg{cases} package.


%
%
\subsection{Matrices}

Like \verb|cases|, matrices are not equation environments on their own,
but can appear inside one.
There are a few variants depending on how you like them:
%
\begin{VerbatimOut}{\jobname.tmp}
\begin{gather*}
\begin{pmatrix}
    1 & 2\\
    3 & 4
\end{pmatrix}\\
\begin{bmatrix}
    1 & \cdots & 1\\
    0 & \ddots & \vdots\\
      & \cdots & 1
\end{bmatrix}\\
\begin{vmatrix}
    a & b\\
    c & d
\end{vmatrix}
\end{gather*}
\end{VerbatimOut}
\ShowExample




%
%
%
\section{Fonts and text in mathematics}

In mathematical mode \TeX{} follows different spacing rules.
Spaces are ignored, and the spacing between letters is slightly altered:
%
\begin{VerbatimOut}{\jobname.tmp}
\emph{different spacing}\\
$different spacing$
\end{VerbatimOut}
\ShowExample
%
To make your documents have that subtle final touch,
it is therefore important to understand where the boundary of text and math goes.
We will first discuss putting \emph{explanatory text} within math environments,
and then talk about \emph{mathematical text} styles like $\exp(f_{\mathrm{stat}})$ below.

\begin{practices}
Also the spacing of punctuation is slightly different in math mode.
You should use math mode only for mathematics;
in particular, the second example below is preferable:
%
\begin{VerbatimOut}{\jobname.tmp}
Summing over $i, j, k$, we find\dots\\
Summing over $i$, $j$, $k$, we find\dots\\
\end{VerbatimOut}
\ShowExample
\end{practices}


%
%
\subsection{Explanatory text}

To put explanatory text inside a math display, use the \cmd{text} command of \pkg{amsmath}.
Since spaces are ignored in math mode, you need to put the spacing inside the argument.
(Quite often, it is useful to add a bit more space with the manual spacing commands described below.)
%
\begin{VerbatimOut}{\jobname.tmp}
\[
a_k < a_{k+1}
\text{ for all } k \in \mathbb N.
\]
\end{VerbatimOut}
\ShowExample
%
The argument inside \cmd{text} is interpreted in text mode,
but it is in fact possible to enter math mode again with \verb|$|:
%
\begin{VerbatimOut}{\jobname.tmp}
\[
a_k < a_{k+1}
\text{ for all $k \in \mathbb N$, and }
a_1 > 1.
\]
\end{VerbatimOut}
\ShowExampleBelow

Inside an \env{align} environment, you can do short textual interjections with \cmd{intertext}.
The benefit is that the alignment is kept in sync across mathematics lines:
%
\begin{VerbatimOut}{\jobname.tmp}
\begin{align}
a^3 - b^3
&= \sum_{k=0}^3 \binom 3 k a^k b^{3-k},\\
\intertext{and by some computation this equals}
&= (a-b)(a^2 + ab + b^2).
\end{align}
\end{VerbatimOut}
\ShowExampleBelow

\begin{practices}
As useful as \cmd{intertext} is,
remember that the display math environments do not participate in page breaking by default.
If you have a long chain of equalities,
you should think about the most reader-friendly way to present it.
\end{practices}


%
%
\subsection{Mathematics fonts}

Then what about mathematics set in a diffenent font?
First, we need to distinguish \emph{operators} and \emph{ordinary symbols} from each other,
since the two also differ in spacing.
Just compare the two examples:
%
\begin{VerbatimOut}{\jobname.tmp}
$\sin x$\\
$\mathrm{sin} x$
\end{VerbatimOut}
\ShowExample
%
The spacing rules around operators are discussed in \Cref{sec:operators}.
The \cmd{operatorname} command described there also takes care of the upright font.

This leaves us finally to talk things like $f_{\mathrm{stat}}$,
$\mathbf R$ or $\mathbb R$, and $\mathcal F$.
The commands are similar to those used in text mode.

The upright roman style is produced with \cmd{mathrm},
bold with \cmd{mathbf},
and the calligraphic style with \cmd{mathcal}.
There is also the sans serif style \cmd{mathsf}.
%
\begin{VerbatimOut}{\jobname.tmp}
$\mathrm{F(x)}$\\
$\mathbf{F(x)}$\\
$\mathsf{F(x)}$\\
$\mathcal{F(x)}$
\end{VerbatimOut}
\ShowExample
%
As you see from above, not all characters are supported by the default fonts,
sometimes producing confusing results.

\begin{practices}
A common convention is to use upright style when something has a multi-letter name.
That is, the identity matrix should be $\mathrm{Id}$ and not $Id$,
and a stationary function should be $f_{\mathrm{stat}}$ and not $f_{stat}$.
\end{practices}

Note that symbols are not typically set in bold font even when \cmd{mathbf} is used:
%
\begin{VerbatimOut}{\jobname.tmp}
\[
\mathbf{\sum_{k=1}^\infty \frac{k}{1+k^2}}
\]
\end{VerbatimOut}
\ShowExample
%
There is the \cmd{boldsymbol} command in \pkg{amsmath},
but it only supports a subset of symbols.
If you need bold mathematics, the easiest way is to use the \pkg{bm} package
and the identically named command:
%
\begin{VerbatimOut}{\jobname.tmp}
% \usepackage{bm}
\[
\bm{\sum_{k=1}^\infty \frac{k}{1+k^2}}
\]
\end{VerbatimOut}
\ShowExample
%
Whenever the font has a special bold symbol, it is used;
otherwise, a ``poor man's bold'' is applied by overprinting the character with small offsets.


By loading the \pkg{amsfonts} package, you gain a couple more styles:
``blackboard bold'' by \cmd{mathbb} and fraktur by \cmd{mathfrak}.
Again, these support only a subset of the characters.
%
\begin{VerbatimOut}{\jobname.tmp}
% \usepackage{amsfonts}
$\mathbb{ABC}$\\
$\mathfrak{ABCdef}$
\end{VerbatimOut}
\ShowExample

In some fields, a blackboard bold $1$ is used (to signify an indicator function, for example).
This symbol is not provided by \pkg{amsfonts}.
There are two methods to fix this issue.

First, $\mathds 1$ can be found e.g.\ in the \pkg{dsfont} package
that provides a double stroke font accessible via \cmd{mathds}.
%
\begin{VerbatimOut}{\jobname.tmp}
% \usepackage{amsfonts, dsfont}
$\mathbb{ABC123}$\\
$\mathds{ABC123}$
\end{VerbatimOut}
\ShowExample
%
Note that the other digits are now missing altogether!

Since \cmd{mathbb} is in muscle memory for many of us, let us show one dirty trick.
If you want to replace the blackboard bold font altogether with the double-stroke font,
you can use the following code:
%
\begin{VerbatimOut}{\jobname.tmp}
% \usepackage{dsfont}
\DeclareCommandCopy{\mathbb}{\mathds}

$\mathbb{ABC123}$
\end{VerbatimOut}
\ShowExample


The second method is more correct,
but it requires a modern \LaTeX{} processing pipeline.
It should thus be used still with caution.

\begin{latexthree}\label{rem:math unicode}
If you copy mathematical text from a \LaTeX-produced PDF,
the symbols come in a normal font or might be completely random.
This is another example of \TeX{} and Unicode having evolved separately.
However, Unicode nowadays has code points for symbols like ``blackboard bold R''.

If you use a Unicode-native compiler (LuaLaTeX or XeLaTeX),
you can load the \pkg{unicode-math} package.
This package replaces all math symbols with their Unicode equivalents.
Now they appear correctly for copy-paste and screen readers.

With this package, you no longer need to load \pkg{amsfonts},
\emph{all digits} have blackboard bold variants,
and there are no longer weird symbol substitutions.
Moreover, it is possible to easily load a different Unicode font for maths.

Where's the catch?
The package is still marked as experimental,
it does not work with pdfLaTeX,
and there are some subtle differences compared to original math commands.
This package should only be used for new projects where backwards compatibility is not required.
\end{latexthree}


%
%
\subsection{Completely different typefaces}

It is possible to use non-default fonts for mathematics,
but there are many caveats.
Most fonts do not contain a usable range of mathematical symbols,
and those that do might not pair well with the text font.

The \pkg{unicode-math} package and Unicode fonts with good mathematics support are the best bet.
Palatino and Noto families also have good mathematics support,
which we will discuss in \Cref{sec:typefaces}.


%
%
\subsection{Units of measure}

The \pkg{siunitx} package simplifies working with numerical quantities and units.
There are a few basic commands that cover most of the cases.

Numbers are formatted with \cmd{num}.
This command accepts exponential notation and error ranges,
and even both \verb|.| and \verb|,| as the decimal separator.
Complex numbers are given with the \cmd{complexnum} command.
%
\begin{VerbatimOut}{\jobname.tmp}
\num{12345}\\
\num{3.9 +- 0,4}\\
\num{1.28e-4}\\
\complexnum{1+2ie3}
\end{VerbatimOut}
\ShowExample

The next basic command is \cmd{unit}, which offers two modes.
In the ``literal'' mode, units are given as strings.
Multiplication is done with \verb|.|, division by \verb|/|,
and sub- and superscripts in the usual way.
In the ``macro'' mode, each part is given as a command
(whose name corresponds to the usual spoken version).
The benefit of this mode is that the appearance can be customized.
%
\begin{VerbatimOut}{\jobname.tmp}
\unit{kg.m.s^{-2}}\\
\unit{\kilogram\metre\per\second\squared}\\
\unit{T_{eff}}
\end{VerbatimOut}
\ShowExample

Finally, the two commands can be put together with \cmd{qty} (or its sibling \cmd{complexqty}):
%
\begin{VerbatimOut}{\jobname.tmp}
\qty{1.7e3}{kg.m/s^2}\\
\qty{17}{\degreeCelsius}\\
\complexqty{1+2ie-3}{\ohm}
\end{VerbatimOut}
\ShowExample
%
For angles, there is a special syntax with \cmd{ang}:
%
\begin{VerbatimOut}{\jobname.tmp}
\ang{60}\\
\ang{12;34;56}
\end{VerbatimOut}
\ShowExample

The package offers a lot of customization in how the quantities are displayed.
(This aligns well with the general \LaTeX{} philosophy.)
See the package documentation for a thorough explanation.

\begin{remark}
See also page~\pageref{ex:table siunitx} for using \pkg{siunitx} in tables.
\end{remark}



%
%
%
\section{Mathematical symbols and whitespace}

\LaTeX{} and \pkg{mathtools} already provide quite a lot of symbols,
but if those are not enough, there are many extension packages.
The first two to consider are \pkg{amssymb} (AMS symbol font)
and \pkg{stmaryrd}.
There are also some specialized packages like \pkg{braket} for Dirac bra-ket notation.

\begin{gotcha}
The \pkg{stmaryrd} package should be loaded \emph{after} \pkg{amssymb},
since it extends and modifies some symbols there.
\end{gotcha}

If you're not interested in browsing through package documentation to find symbols,
the Detexify tool\footnote{\url{detexify.kirelabs.org}} is your friend.\index{Detexify}\index{symbols!finding}
You can draw the symbol in this web app,
and it searches for it in \LaTeX{} symbol list and a large collection of extension packages.%
\footnote{\url{www.ctan.org/tex-archive/info/symbols/comprehensive}
is the source if you really want to search manually;
this PDF is over 30~MB and almost 500~pages.}


%
%
\subsection{Spacing}\label{sec:math spacing}

\TeX{} is quite smart about figuring out the spacing between different mathematical symbols.
Just look at the difference between $2 - 1$ and $2 (-1)$.
Sometimes, it is important to choose the command properly to get expected spacing.
Compare the two examples:
%
\begin{VerbatimOut}{\jobname.tmp}
$f : A \to B$\\
$f \colon A \to B$
\end{VerbatimOut}
\ShowExample

Let us first see the manual commands for spacing,
since their sizes correspond to units used by \TeX.
Their sizes are defined based on the \emph{math unit}, which depends on the font size.
Note that the difference of the units is very small,
just enough to be perceptible:\indexcmd{mspace}
%
\begin{VerbatimOut}{\jobname.tmp}
$xx$ (no space)\\
$x\,x$ (thin space, 3~mu)\\
$x\:x$ (medium space, 4~mu)\\
$x\;x$ (thick space, 5~mu)\\
$x\!x$ (negative thin space)\\
$x\mspace{8mu}x$ (custom space)
\end{VerbatimOut}
\ShowExample
%
These commands can be used to manually tweak spacings.
However, even better is to let \TeX{} automate things.


Internally, every mathematical symbol belongs to a symbol class.\index{symbol classes}
These are listed in \Cref{tbl:math symbol classes}.
\TeX{} puts a space between two symbols based on their respective classes.
For example, there is no space between two ordinary characters,
whereas there is a medium \verb|\:| space between binary and ordinary symbols.

\begin{table}[tbh]
\centering
\begin{tabular}{l|cc}
Symbol class & Override command & Examples\\
\hline
Ordinary & \cmd{mathord} & $2 x$\\
Operator & \cmd{mathop} & $\sin$\\
Binary & \cmd{mathbin} & $2 + x$\\
Relation & \cmd{mathrel} & $2 < x$\\
Opening & \cmd{mathopen} & $( \quad \lfloor$\\
Closing & \cmd{mathclose} & $) \quad \rfloor$\\
Punctuation & \cmd{mathpunct} & $,$
\end{tabular}
\caption{The mathematical symbol classes.}
\label{tbl:math symbol classes}
\end{table}

We talk more about operators below in \Cref{sec:operators},
so let us consider binary operations as our example.
Usually $\heartsuit$ is considered an ordinary symbol,
but let us imagine that we have defined it as an operation between two expressions.
Then it would be necessary to wrap its use in \cmd{mathbin} in order to get correct spacing.
%
\begin{VerbatimOut}{\jobname.tmp}
\newcommand{\friends}{\mathbin\heartsuit}

Bad: $x \heartsuit y$\\
Good: $x \friends y$.
\end{VerbatimOut}
\ShowExample
%
Some combinations of symbol classes are considered impossible,
in which case \TeX{} modifies one of the classes suitably.
To go back to the example of the minus sign,
\verb|-| is usually defined as a Relation symbol.
However, if it is not preceded by an Ordinary symbol (like a number),
it is transformed into Ordinary itself -- this means that \verb|-1| produces no spacing at all.

\begin{gotcha}
This somewhat explains the difference between \verb|:| and \cmd[spacing]{colon}, but only partly.
The symbol \verb|:| is classified as Relation and \cmd[spacing]{colon} as Punctuation.
However, \pkg[colon]{amsmath} further modifies the spacing of \cmd[spacing]{colon}
to fit the $f \colon A \to B$ pattern -- and only that pattern.

If you ever need the colon as a punctuation symbol, you can wrap it in \verb|\mathpunct{:}|.
\end{gotcha}

\begin{gotcha}
Another source of confusion is \verb.|. and \cmd[spacing]{mid}.
The first is Ordinary and the second Relation.
They should be used correspondingly:
\verb.|a|. for the absolute value $|a|$,
\verb.f|_A. for the restriction of a function $f|_A$,
and \verb.a \mid b. for the divisibility relation $a \mid b$.
\end{gotcha}

\begin{gotcha}
The symbols \verb|. ! ?| are of class Ordinary, not Punctuation.
This is to do with their mathematical meanings.
\end{gotcha}

\begin{gotcha}
If you write in a language where comma is used as the decimal separator,
you should wrap the comma in parentheses:
%
\begin{VerbatimOut}{\jobname.tmp}
$123,456$\\
$123{,}456$
\end{VerbatimOut}
\ShowExample
%
It is also possible to load the \pkg{icomma} package,
which interprets comma as a list separator \emph{only if} it is followed by a space.
\end{gotcha}

\begin{gotcha}
Adding an accent turns the symbol into Ordinary.
%
\begin{VerbatimOut}{\jobname.tmp}
$a \hat= b$\\
$a \mathrel{\hat=} b$
\end{VerbatimOut}
\ShowExample
\end{gotcha}

\begin{gotcha}
The occasionally used notation for open intervals conflicts with the algorithm,
since \verb|]a, b[| is understood as Closing--Opening pair instead of the opposite.
It can be fixed manually as in the next example:
%
\begin{VerbatimOut}{\jobname.tmp}
$f \colon ]a, b[ \to ]0, 1]$\\
$f \colon \mathopen]a, b\mathclose[
    \to \mathopen]0, 1]$
\end{VerbatimOut}
\ShowExample
%
A quick fix is to wrap the intervals inside \verb|{}|.
If you use a lot of intervals,
you should either define some custom commands or check out the \pkg{interval} package.
\end{gotcha}



One common pain point is the humble $\mathrm dx$ that appears in integrals.
It is nice to get the spacing consistently right,
and note also the upright $\mathrm d$ (this is advocated by an ISO standard, but some prefer $dx$).
\index{dx@$\mathrm dx$}

A common way to do this (based on Stack Exchange discussions) is:
%
\begin{VerbatimOut}{\jobname.tmp}
\newcommand{\diff}{\mathop{}\!\mathrm{d}}

In an integral:
\[
\int_0^1 x \diff x,
\]
and inside parens: $\mu(\diff x)$.
\end{VerbatimOut}
\ShowExample
%
Note the automatic space between $x$ and $\mathrm dx$.

What happens here is that the empty group \verb|{}| is set in operator style.
When preceded by an ordinary symbol, the operator style puts a thin \verb|\,| space;
when preceded by punctuation like \verb|(|, the space is not present.
The thin \verb|\,| space between the (empty) operator and `d' is cancelled with \verb|\!|.

If you find yourself writing lots of complicated differentials like
\[
\frac{\mathrm d^2 \log(x)}{\mathrm dx^2},
\]
you should look into the \pkg{diffcoeff} package
that provides a slightly shorter syntax.


%
%
\subsection{Operators and limits}\label{sec:operators}
One common class of text in mathematics is operators.
Compare the three examples here:
%
\begin{VerbatimOut}{\jobname.tmp}
Awful: $2 sin \pi$,\\
Still bad: $2 \textrm{sin} \pi$,\\
Good: $2 \sin \pi$.
\end{VerbatimOut}
\ShowExample
Here the \verb|\sin| command not only sets the operator name properly,
but also adjusts the spacing around the operator.

If the operator you need is not predefined as a \LaTeX{} command,
you can do the styling once with \cmd{operatorname}\dots
%
\begin{VerbatimOut}{\jobname.tmp}
\[
2 \operatorname{dim} X
\]
\end{VerbatimOut}
\ShowExample
%
\dots{}or define a new command with \cmd{DeclareMathOperator} in the preamble:
%
\begin{VerbatimOut}{\jobname.tmp}
% In the preamble:
% \DeclareMathOperator{\arccosh}{arccosh}

\[
2 \arccosh \pi
\]
\end{VerbatimOut}
\ShowExample

For certain operators, limits can be put either to the side or above/below the operator.
Compare these two:
\[
\text{text style }\textstyle \sum_{i=1}^\infty
\text{ and display style }\displaystyle \sum_{i=1}^\infty.
\]
There are package options to \pkg{amsmath}/\pkg{mathtools}
to control how the limits are placed in display equations.
By default, they are placed on the side for integrals and above/below for everything else.

To put limits above/below your custom operator, use the starred form of the declaration:
%
\begin{VerbatimOut}{\jobname.tmp}
\[
\operatorname{dim}_H^\ast X
\text{ vs }
\operatorname*{dim}_H^\ast X
\]
\end{VerbatimOut}
\ShowExample

If you need to put multiple lines of text in a sum argument,
you can use the \cmd{substack} command.
Inside its argument, you can use the usual \verb|\\| line breaks:
%
\begin{VerbatimOut}{\jobname.tmp}
\[
\sum_{\substack{1 \leq j \leq 10\\ 1 \leq k \leq j}}
\]
\end{VerbatimOut}
\ShowExample

\todo{mathtools offers cramped versions}

Some special operators that deserve a mention are the modulus operators,
since they have some special spacing rules:
%
\begin{VerbatimOut}{\jobname.tmp}
\begin{gather*}
a \equiv b \mod p\\
a \equiv b \bmod p\\
a \equiv b \pmod p
\end{gather*}
\end{VerbatimOut}
\ShowExample


%
%
\subsection{Fractions}
There are two basic fraction-like commands: \cmd{frac} for fractions,
and \cmd{binom} for binomial coefficients.
Both take the numerator and denominator as their arguments.

To produce generalized fraction-like operators like
\[
\genfrac{(}{)}{}{}{a}{b}
\quad\text{and}\quad
\genfrac{\lfloor}{\rceil}{0pt}{}{a}{b},
\]
the \pkg{amsmath} package provides the \cmd{genfrac} command.
Read the package documentation to understand its six parameters.

Continued fractions are produced with the \cmd{cfrac} command:
%
\begin{VerbatimOut}{\jobname.tmp}
\[
\cfrac{1}{\sqrt 3 + \cfrac{2}{\sqrt 3 + \dotsb}}
\]
\end{VerbatimOut}
\ShowExample

\begin{warning}
Plain \TeX{} does fractions and binomials with the \obscmd{over} and \obscmd{choose} commands
that have a different syntax: \verb|n \choose m| instead of \verb|\binom n m|.
As natural as the syntax might seem, you should only use the \LaTeX{} constructs.
\end{warning}


%
%
\subsection{Building new symbols}

\todo{Shortly on this}

\todo{Negations}



%
%
%
\section{Size in mathematics}

Delimiting symbols come in five sizes:
\begin{VerbatimOut}{\jobname.tmp}
\[
( \big( \Big( \bigg( \Bigg(
\]
\end{VerbatimOut}
\ShowExample
%
Instead of setting the size manually,
the \cmd{left} and \cmd{right} commands can be used.
They choose the size based on the content between delimiters.
Accordingly, the two commands must be paired correctly.
There is also \cmd{middle} for an optional middle delimiter in the matching size.
%
\begin{VerbatimOut}{\jobname.tmp}
\[
\left\{ a \in \mathbb Q
  \middle| a^2 < \frac 1 2 \right\}
\]
\end{VerbatimOut}
\ShowExample

\begin{gotcha}
It is not possible to have a line break between \cmd{left} and \cmd{right},
so in long formulas you might need to do the sizing manually.
\end{gotcha}

\begin{practices}
I like to define the following two commands in my documents:
%
\begin{VerbatimOut}{\jobname.tmp}
\newcommand{\abs}[1]{\left|#1\right|}
\newcommand{\norm}[1]{\left\|#1\right\|}

Absolute value $\abs x$,\\
function norm $\norm f$.
\end{VerbatimOut}
\ShowExample
%
They are both semantically meaningful and automatically sized.
\end{practices}

There are two basic styles of sizing other mathematics: text style and display style
(there are also two levels of sub-/superscript styles).
The first is used with inline mathematics and the latter with display mathematics.
It is possible to override the style with \cmd{textstyle} and \cmd{displaystyle}:
%
\begin{VerbatimOut}{\jobname.tmp}
Text with oversize
$\displaystyle \binom n m \frac 1 2$
mathematics, and a display with small maths:
\[
\sqrt{\textstyle \binom n m \frac 1 2}
\]
\end{VerbatimOut}
\ShowExample
%
For fractions, there are the abbreviated \cmd{tfrac} and \cmd{dfrac} commands
to achieve this effect.


%
%
%
\section{Decorations}

\subsection{Accents}

You should not use accented Unicode characters like `â' in math mode.
Instead, write \verb|\hat a|.
The supported accents are listed in various places online.%
\footnote{\url{https://en.wikibooks.org/wiki/LaTeX/Mathematics\#Accents}}

\begin{gotcha}
The accents apply to the immediately succeeding character,
and do not take subscripts into account.
Compare the three:
%
\begin{VerbatimOut}{\jobname.tmp}
$\hat a_0$ vs $\hat {a_0}$
vs $\widehat {a_0}$
\end{VerbatimOut}
\ShowExample
\end{gotcha}

To put material over and under symbols,
there are the aptly named \cmd{overset}, \cmd{underset}, and \cmd{overunderset} commands:
%
\begin{VerbatimOut}{\jobname.tmp}
\[
\overset{\eqref{eq:pythagoras}}{=}
\quad \underset{\ast}{X}
\quad \overunderset{a}{b}{C}
\]
\end{VerbatimOut}
\ShowExample
%
To produce some very cursed versions of big operators,
there is also the \cmd{sideset} command:
%
\begin{VerbatimOut}{\jobname.tmp}
\[
\sideset{^a_b}{^c_d}\sum_{i=1}^\infty
\]
\end{VerbatimOut}
\ShowExample



%
%
\subsection{Dots}

The \pkg{amsmath} package provides a context-sensitive \cmd{dots} command.
Compare the vertical positions of the dots in these two examples:
%
\begin{VerbatimOut}{\jobname.tmp}
\begin{gather*}
1 + \dots + n\\
1, \dots, n
\end{gather*}
\end{VerbatimOut}
\ShowExample
%
Thanks to this, is is usually enough to use \cmd{dots} everywhere in your mathematics.
The exception is at the end of expression,
since \pkg{amsmath} decides the placement by looking at the following symbol.
You can give a hint with commands like \verb|\dotsc| (dots with commas)
and \verb|\dotsb| (dots with binary operations)
and even \verb|\dotsi| (dots with integrals) for these cases.

Of course, if memorizing these seems too much, the usual \verb|\cdots| and \verb|\ldots| can be used;
they just tie the semantics and presentation together.


%
%
\subsection{Braces and highlighting}
There are two commands for producing horizontal braces:
\cmd{overbrace} and \cmd{underbrace}.
They both accept explanatory text as super-/subscript respectively.
%
\begin{VerbatimOut}{\jobname.tmp}
\[
\overbrace{a^2 + b^2}^{\text{catheti}}
= \underbrace{c^2}_{> 0}.
\]
\end{VerbatimOut}
\ShowExample

The \pkg{mathtools} package also provides similar commands for brackets.
They can be optionally customized in thickness and height,
but they do not support text:
%
\begin{VerbatimOut}{\jobname.tmp}
\[
\overbracket[1pt][0.5cm]{a^2 + b^2}
= \underbracket[3pt]{c^2}.
\]
\end{VerbatimOut}
\ShowExample

Relatedly, let us just note that \pkg{amsmath} provides a \cmd{boxed} command
for drawing a box around some mathematics.
If the argument continues across a \verb|&| alignment point,
you need the \pkg{mathtools} variant \cmd{Aboxed}.
A more extensible version is provided by the \pkg{empheq} package.
%
\begin{VerbatimOut}{\jobname.tmp}
\[
\boxed{a^2 + b^2} = c^2.
\]
\end{VerbatimOut}
\ShowExample

If you want to emphasize something with colour,
the \pkg{xcolor} package offers a \cmd{mathcolor} command that works
just like its text counterpart:%
\footnote{To be precise, it works \emph{very much unlike} its text counterpart in that
if you try to use \texttt{\textbackslash textcolor} in math mode, it breaks the spacing.}
%
\begin{VerbatimOut}{\jobname.tmp}
\[
\mathcolor{blue}{a^2 + b^2} = c^2.
\]
\end{VerbatimOut}
\ShowExample


%
%
\subsection{Arrows}
Mathematicians seem to like arrows.\index{arrows!mathematics}
Here is a sample of some;
note that some have synonyms for semantically more meaningful code.
You probably can deduce the rules for even further variants:
\begin{VerbatimOut}{\jobname.tmp}
\begin{gather*}
\gets \to \mapsto\\
\leftarrow \uparrow \nwarrow \searrow \rightarrow\\
\Leftarrow \Leftrightarrow \Rightarrow\\
\Longleftarrow \Longleftrightarrow \implies\\
\hookleftarrow \rightharpoonup\\
\leftrightharpoons \rightleftharpoons
\end{gather*}
\end{VerbatimOut}
\ShowExample

Sometimes it is useful to put longer sub- or superscripts on arrows.
The \pkg{amsmath} package provides extensible versions of arrows.
The command names are prefixed with \verb|x|,
and the optional argument goes below and the main argument above the arrow:
%
\begin{VerbatimOut}{\jobname.tmp}
\begin{gather*}
f(x) \xrightarrow[n \to \infty]{\text{weakly}} 0\\
b^2 = 0
\xRightarrow{\eqref{eq:pythagoras}} a^2 = c^2
\end{gather*}
\end{VerbatimOut}
\ShowExample

\begin{gotcha}
These extended arrows should only be used in display size.
\end{gotcha}

While we are on the topic of arrows,
let us mention the commutative diagrams that are popular among some mathematicians.
See \Cref{sec:tikz-cd} for a solution based on TikZ.



%
%
%
\section{Theorem environments}\label{sec:amsthm}

Basic theorem/definition/etc.\ environments are usually done with the \pkg{amsthm} package.
The operative command is \cmd{newtheorem}, which takes two required arguments:
the environment name and the heading to display.
It also takes one optional argument, whose location is significant:
\begin{itemize}
\item If it comes after the required arguments,
    it gives the level at which the environment is numbered.
    For example, \verb|[section]| means that the counter is prefixed with section number.
\item If it comes in between them, it gives the counter to use.
    If you have already defined a theorem environment named \verb|definition|,
    then \verb|[definition]| means that the new environment is part of the same numbering.
\end{itemize}

\begin{practices}
This is now more of a personal opinion,
but I strongly believe that all theorem-like environments should share the same numbering.
It is infuriating to find Theorem~5 in a document like
\begin{quote}
    \dots, Definition~3, Lemma~3, Theorem~3, Lemma~4, Lemma~5, Lemma~6, Theorem~4,
    Definition~4, \dots
\end{quote}
where each environment has individual numbering.
\end{practices}

The appearance of the environment is chosen with the \cmd{theoremstyle} command.
It applies to all following \cmd{newtheorem} invocations.
There are three predefined styles:
\begin{description}
\item[plain] Boldface header, italic text.
\item[definition] Boldface header, upright text.
\item[remark] Italic header, upright text.
\end{description}
%
There is also a \cmd{swapnumbers} command if you prefer ``1.1~Theorem'' instead of ``Theorem~1.1''.

A complete example of defining theorem environments could be as follows.
All environments share the same numbering, which is section-based.
%
\begin{ExampleCode}
\theoremstyle{plain}
\newtheorem{theorem}{Theorem}[section]
\newtheorem{lemma}[theorem]{Lemma}
\newtheorem{corollary}[theorem]{Corollary}

\theoremstyle{remark}
\newtheorem{remark}[theorem]{Remark}
\newtheorem{example}[theorem]{Example}

\theoremstyle{definition}
\newtheorem{definition}[theorem]{Definition}
\end{ExampleCode}

There is also the starred \cmd{newtheorem*} command that does not produce a number.
It might be useful in some special cases.
%
\begin{VerbatimOut}{\jobname.tmp}
\theoremstyle{plain}
\newtheorem*{riemannhyp}{Riemann hypothesis}

\begin{riemannhyp}
The real part of every nontrivial zero of the Riemann zeta function is $1/2$.
\end{riemannhyp}
\end{VerbatimOut}
\ShowExampleBelow


The theorem environment also accepts an optional argument,
which is placed in parentheses after the theorem number.
%
\begin{VerbatimOut}{\jobname.tmp}
\begin{theorem}[Pythagoras]
If $a$ and $b$ are the lengths of catheti of a right-angled triangle,
then the hypothenuse has length $\sqrt{a^2 + b^2}$.
\end{theorem}
\end{VerbatimOut}
\ShowExampleBelow


The \pkg{amsthm} package automatically defines a \env{proof} environment.
The ``Proof'' heading can be customized with the optional argument,
and the environment automatically put a QED symbol at the end.
The symbol can be changed by redefining \cmd{qedsymbol}
(either in the preamble or locally within the environment):\label{ex:nasty qed hack}
%
\begin{VerbatimOut}{\jobname.tmp}
% We need to fake things a bit since the notes use ntheorem instead of amsthm
% To use a customized symbol, we need to call \qed at the closing of of the environment
% Moreover, instead of \renewcommand\qedsymbol{...} we need to CALL \qedsymbol{...}
% Instead of showing and executing different code, we use a nasty hook here
\DeclareCommandCopy{\oldqedsymbol}{\qedsymbol}
\AddToHook{env/proof/end}[qed-hack]{%
\renewcommand{\qedsymbol}{\oldqedsymbol}\qedsymbol{$\heartsuit$}\qed}

\begin{theorem}
\LaTeX{} course participants are fantastic.
\end{theorem}
\begin{proof}[Sketch of proof]
Just look at them.
\renewcommand{\qedsymbol}{$\heartsuit$}
\end{proof}
\end{VerbatimOut}
\ShowExampleBelow[9]
\RemoveFromHook{env/proof/end}[qed-hack]

If the environment ends with a list or display equation,
the QED symbol is placed on a new line.
Since this is not very pretty, we can give \LaTeX{} a hint with the \cmd{qedhere} command.
%
\begin{VerbatimOut}{\jobname.tmp}
% Again a hack, since ntheorem automatically puts the QED mark in the correct position
\newcommand{\qedhere}{}

\begin{proof}[Remainder of a long proof]
It remains to notice that
\[
a^2 - b^2 = (a+b)(a-b).
\qedhere
\]
\end{proof}
\end{VerbatimOut}
\ShowExampleBelow[3]


There are also some fancier theorem environments,
such as those produced by \pkg{ntheorem}.
However, \pkg{ntheorem} does have some subtly surprising behaviors at times,
and it might have compatibility issues with some other packages.%
\footnote{This observation sponsored by: using \pkg{ntheorem} in quite a few projects,
including these notes.
Since the examples use \pkg{amsthm} syntax and \pkg{ntheorem} is at times very incompatible with it,
there are some nasty hacks involved.}

\emph{The \LaTeX{} Companion} \cite[Section~4.1.5]{TLC} suggests using the \pkg{thmtools}
package to iron out some differences between \pkg{amsthm} and \pkg{ntheorem},
but I have no personal experience of it.


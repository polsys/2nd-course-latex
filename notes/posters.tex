\chapter{Posters with tikzposter}

There are several methods to creating posters with \LaTeX{}.
To mention just three of the many packages: \pkg{tikzposter}, \pkg{beamerposter}, and \pkg{tcolorbox}.
All of them suffer from the fact that posters are very visual,
but \TeX{} can really only do boxes.

Here we discuss \pkg{tikzposter}, since it is reasonably simple, reliable, and customizable.
However, it is worth noting that the package has not been actively maintained since~2014.
The other packages work fairly similarly.


%
%
\section{Basic layout}

The package is loaded with
%
\begin{ExampleCode}
\documentclass{tikzposter}
\end{ExampleCode}
%
By default, the options \verb|a0paper|, \verb|25pt|, and \verb|portrait| are used.
Paper sizes A1 and A2 are also available, and the \verb|landscape| option produces a landscape page.
The range of font sizes is limited to 12, 14, 17, 20, and 25~points.
Of these, only the last one is suitable for A0~size.

\begin{practices}
As with presentations, you should not have too much text.
Use whitespace and structure to guide the reader.
\end{practices}

For one reason or other, the package adds a watermark to the lower-right corner.
It can be turned off with the monstrously long command
%
\begin{ExampleCode}
\tikzposterlatexaffectionproofoff
\end{ExampleCode}
%
There are also some package options to control margins and default block spacing;
see the documentation for details.

The title is set with \cmd[tikzposter]{maketitle} as usual;
you can set \cmd{title}, \cmd{author}, and \cmd{institute} in the preamble.
You can also put graphical (or rather, any) commands inside \cmd{titlegraphic}.
By default these elements are set on top of each other.
The customization is briefly discussed in \Cref{sec:poster style}



As with Beamer, the basic layout consists of blocks.
The \cmd[tikzposter]{block} command takes optional customization arguments,
a title, and the contents:
%
\begin{ExampleCode}
\block[]{Block title goes here}
{
Contents of the block
}
\end{ExampleCode}
%
The block will fill the available horizontal space,
and the vertical size depends on the contents.
Contents can be any material (except floats).
If the title is empty, the ``block header'' disappears.

To set several blocks side by side, it is possible to use columns
specified inside a \env{tikzposter}{columns} environment.
A new column is started with the \cmd[tikzposter]{column} command.
As its argument it takes the fraction of text area width.%
\footnote{You need to ensure that they sum to at most $1$.
The documentation also talks about absolute widths, but they do not seem to actually work.}
Any blocks specified after this command will be placed vertically in the column.
For example, here we set two columns of equal size,
and place two blocks in the left column:
%
\begin{ExampleCode}
\begin{columns}

\column{0.5}
\block{Above left}{}
\block{Below left}{}

\column{0.5}
\block{Right}{}

\end{columns}
\end{ExampleCode}

The \env[tikzposter]{subcolumns} and \cmd[tikzposter]{subcolumn}
can be used to further divide a column
(the arguments are fractions of the \emph{outer column} width).

After the \env{columns} environment ends,
all new blocks again fill the whole horizontal space.

You cannot use the usual vertical spacing commands to adjust spacing between blocks.
A workaround is to add an offset both to the block title and contents.
Negative $y$ values point downwards.%
\footnote{It appears that this method might not work with blocks without title;
this seems to be a bug.}
%
\begin{ExampleCode}
\block[titleoffsety=-2cm, bodyoffsety=-2cm]{Title}{Contents}
\end{ExampleCode}

Inside a block, you can use the \cmd[tikzposter]{coloredbox} command to create highlighted regions.
%
\begin{ExampleCode}
\block{Goal}
{
In this poster we discuss Pythagoras' theorem.

\coloredbox{It is maybe the most important theorem in human history!!}

\coloredbox[bgcolor=PeachPuff, fgcolor=red]{\centering I mean it!!!}
}
\end{ExampleCode}
%
It is also possible to add small notes.
They are defined outside blocks, but they attach to the previous block.
By default, they connect to the center of the block,
but the precise point can be adjusted with optional arguments,
as with the distance and angle (here east-southeast) to the anchor point:
%
\begin{ExampleCode}
\note[angle=-20, radius=8cm, connection, targetoffsety=-2cm]
    {This poster will convince you about it!}
\end{ExampleCode}

Let us see how these pieces (and a few more content additions) play together.
The output of the following code is shown in \Cref{fig:poster example}.

\begin{figure}
\centering
\includegraphics[height=0.85\textheight]{examples/tikzposter-content.pdf}
\caption{The example poster (set in A2 size, and scaled here).}\label{fig:poster example}
\end{figure}

\VerbatimInput[fontsize=\footnotesize,frame=lines]{examples/tikzposter-content.tex}




%
%
\section{Styling}\label{sec:poster style}

The styling system of \pkg[styling]{tikzposter} is very similar to that of Beamer.
There are a few built-in styles.%
\footnote{\url{https://bitbucket.org/surmann/tikzposter/downloads/styleguide.pdf}}


\begin{technote}
The source code for the predefined styles is very readable.
If you want to customize styles, I recommend using their code as a starting point.

The installation location of packages depends on your specific \LaTeX{} installation,
but the code can also be browsed online at \url{https://bitbucket.org/surmann/tikzposter/}.
\end{technote}


The overall look is controlled by \cmd[tikzposter]{usetheme},
and then e.g.\ the colour and background styles can be customized.
Individual colours can be changed with \cmd[tikzposter]{colorlet}
(unlike in Beamer, colours are not separated into foreground and background colours).

For example, this code produces \Cref{fig:poster styled}:
\begin{ExampleCode}
\usetheme{Envelope} % Block and title style set by this
\usecolorstyle{Denmark} % Red-white colour scheme

\usebackgroundstyle{VerticalGradation} % Vertical gradient...
\colorlet{backgroundcolor}{blue!40} % ...with a faint blue colour
\end{ExampleCode}

To modify the layout of the title contents,
it is unfortunately necessary to go to low-level \LaTeX{} code.
The \cmd[tikzposter]{settitle} command can be used to define the layout code for the title contents.
The variables set by the \verb|\maketitle| companion commands can be used as below.
The following code is also used in \Cref{fig:poster styled}:
%
\begin{ExampleCode}
\settitle{
    \makebox[8cm][c]{\@titlegraphic}
    \parbox[b]{20cm}{
        \color{titlefgcolor} {\bfseries \Huge \sc \@title \par}
        \vspace*{1em}
        {\huge \@author{} \Large (\@institute)}
    }
}
\end{ExampleCode}

\begin{figure}
\centering
\includegraphics[height=0.85\textheight]{examples/tikzposter-style.pdf}
\caption{A styled version of the previous example.}\label{fig:poster styled}
\end{figure}


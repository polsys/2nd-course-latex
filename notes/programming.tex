\chapter{Programming}

What really is \LaTeX?

It all starts with \TeX, the typesetting system designed by Donald Knuth in late 70s and 80s.
\TeX{} is a powerful system for laying out text,
but it puts a lot of responsibility on the writer.
Essentially, you are responsible for both the content and its presentation.

\LaTeX{} (originally by Leslie Lamport, now maintained by several people)
extends this core by introducing commands like \cmd{section}:
this command not only lays out the section title in a bigger font,
but also adds some vertical whitespace around it, updates the current section number,
adds an entry to the table of contents, and so on.
This \emph{separates the content from how it is presented}.

We will try to emphasize this philosophy on this course.
By using standard constructs, packages, and our own custom commands,
we can make the \verb|.tex| source code \emph{semantically} meaningful:
something that you could read aloud.
The system then makes sure that the output looks great on paper
-- isn't that why we ever wanted computers to exist!


\begin{warning}
Even when using \LaTeX{}, the old \TeX{} is still there.
It is possible to use plain-\TeX{} commands in your documents,
but \emph{this is highly discouraged}.
These commands sidestep all the enhancements brought by \LaTeX,
and might cause hard-to-diagnose problems.
You might need them when writing more complicated packages,
but then you are already solidly in the ``you are free to shoot your own foot'' territory.
Some such commands are noted in these Warnings.

Unfortunately, many heritage templates and Stack Exchange answers still mix
\TeX{}, \LaTeX, and pre-1994 \LaTeX.
Examples of the latter include the obsolete \obscmd{em}, \obscmd{bf}, and \obscmd{it} commands
(e.g.\ writing \verb|{\bf bold}| instead of \verb|\textbf{bold}|).
\end{warning}


\begin{latexthree}
The ``standard'' \LaTeX{} version is $2\varepsilon$, released in 1994.
\LaTeX3 is a long-running project (started already before $2\varepsilon$, and reactivated around 2018)
on rewriting the internals of \LaTeX{} to be more extensible.

\LaTeX3 introduces new commands for defining commands and environments.
We will only discuss the version $2\varepsilon$ methods (which are still valid) here,
but if you find yourself writing a package, you should consider learning about the new methods.

We will talk about some new features and accessibility improvements brought by \LaTeX3 below,
starting already in \Cref{sec:document structure}.
\end{latexthree}



%
%
%
\section{Commands and environments}

\subsection{Defining commands}
\index{macros}
\TeX{} is a macro language.
This means that there are special source code tokens (\emph{macros}),
that are \emph{expanded} into sequences of more tokens
(including more macros, which are expanded recursively).
The core system includes only a few commands for manipulating internal state and putting out text,
and the rest is achieved via macro expansion.

That was a mouthful, so let us see an example.
In \LaTeX{} macros are defined with the \cmd{newcommand} command.

\begin{VerbatimOut}{\jobname.tmp}
\newcommand{\hello}{Hello world!}

\hello
\end{VerbatimOut}
\ShowExample

Here \cmd{newcommand} takes two arguments:
the first is the name of the macro to define, and the second is what the macro expands to.
In \TeX{} things are grouped with the \verb|{}| braces.
(The first set of braces is technically unnecessary, but it looks cleaner in my opinion.)

Let us see another example where the command is used repeatedly,
and also expanded inside other commands:
\indexcmd{MakeUppercase}
\begin{VerbatimOut}{\jobname.tmp}
\newcommand{\hello}{Hello world}

--\hello, they whispered.\\
--I said, \MakeUppercase{\hello}!
\end{VerbatimOut}
\ShowExample

Since the macros are expanded recursively until there is nothing more to expand,
you can write things that are semantically very unclear,
but comprehensible to a computer:
\begin{VerbatimOut}{\jobname.tmp}
\newcommand{\hi}{^\infty}
\newcommand{\underscore}{_}
\newcommand{\lo}{\underscore{i=1}}

\[ \sum\lo\hi \]
\end{VerbatimOut}
\ShowExample
%
The expansion goes something like this:
\begin{enumerate}
    \item \verb|\sum\lo\hi|
    \item \verb|\sum\underscore{i=1}\hi|
    \item \verb|\sum_{i=1}\hi|
    \item \verb|\sum_{i=1}^\infty|
\end{enumerate}
So the end result is the same as if one had written the sum sensibly from the beginning.



\begin{warning}
In plain \TeX{} macros are defined with \obscmd{def} and \obscmd{let}.
If you see code using either,
you have entered the ``shoot your own foot'' territory.\footnotemark
\end{warning}
\footnotetext{Since you asked, the difference between the two is whether the contents of the macro
are expanded at usage or definition time.
Neither protects you against overwriting existing macros.}
\quiettodo{Keep these on the same page!}


%
\subsection{Arguments to commands}

To pass arguments to macros, we can indicate their number with an optional argument
between the macro name and definition.
These arguments can then be accessed in the macro definition with \verb|#1|, \verb|#2|, and so on.

\begin{VerbatimOut}{\jobname.tmp}
\newcommand{\say}[2]{#1 says ``#2''!}

\say{Petri}{Hi}
\end{VerbatimOut}
\ShowExample

Arguments are often wrapped in the \verb|{}| braces,
but it is not actually necessary -- in a very specific case.
By default, \LaTeX{} interprets a single letter or a number as an argument.
The braces extend the argument to a longer stretch.

This means that all of the following are equivalent:

\begin{VerbatimOut}{\jobname.tmp}
\[
\frac{1}{2}
= \frac 1 2
= \frac12.
\]
\end{VerbatimOut}
\ShowExample

Do note the last one -- \LaTeX{} command names can only consist of letters,
so the number 1 is not interpreted as part of the command name but an argument.
There is no requirement to separate the name and a following number.
(I do find the last example hard to read, though.)
Conversely, whitespace won't separate longer arguments; you really need the braces:

\begin{VerbatimOut}{\jobname.tmp}
\[
\frac 12 34
\neq \frac {12} {34}.
\]
\end{VerbatimOut}
\ShowExample

However, commands are interpreted as single tokens.
This happens regardless of whether the command would expand to several tokens.

\begin{VerbatimOut}{\jobname.tmp}
\newcommand{\magic}{314 159}
\[ \frac \magic \pi \]
\end{VerbatimOut}
\ShowExample

\begin{gotcha}
Some built-in macros swallow the whitespace that follows them.
If a word space is actually needed, you can feed the command an empty group \verb|{}|:
\begin{VerbatimOut}{\jobname.tmp}
\LaTeX programming\\
\LaTeX{} programming
\end{VerbatimOut}
\ShowExample
\end{gotcha}


%
\subsection{Groups and scopes}

\index{group}\index{scope}
More precisely, the braces delimit a \emph{group}.
A group is handled as a single argument to a command.
Additionally, they serve as a \emph{scope} for commands that change
how all the following text is output.
For example, the font-sizing commands like \cmd{tiny} and \cmd{Large}
affect the size of all text that follows them.
However, this effect only lasts until the group is closed with a \verb|}|.

\begin{VerbatimOut}{\jobname.tmp}
{\tiny I am small}
and I am normal
and {\Large I am large}
\end{VerbatimOut}
\ShowExample

If you forget about this and just write \cmd{tiny} without any scoping,
you will have tiny text until the end of document (or the next font size command).

\begin{VerbatimOut}{\jobname.tmp}
\tiny I am small and
{\Large I am large}
and I am tiny again
\end{VerbatimOut}
\ShowExample

There is one place where the braces do not delimit a group:
in the definition of commands.
In the following example, \verb|\mouse{Squeak}| is expanded into
\verb|\tiny Squeak| and not into \verb|{\tiny Squeak}|.
Therefore the effect persists onto the following line.
%
\begin{VerbatimOut}{\jobname.tmp}
\newcommand{\mouse}[1]{Mouse: \tiny#1.}

\mouse{Squeak}\\
I: Oops.
\end{VerbatimOut}
\ShowExample
%
The scope can be introduced here by adding \verb|{}| into the command definition:
%
\begin{VerbatimOut}{\jobname.tmp}
\newcommand{\mouse}[1]{Mouse: {\tiny#1}.}

\mouse{Squeak}\\
I: Alright.
\end{VerbatimOut}
\ShowExample

Scopes also affect things like \cmd{newcommand}:
the command \verb|\mouse| defined above only exists
within the example block, and is not accessible beyond it.
\todo{I want an optional cmd argument for subindex}



%
\subsection{Optional arguments}

Some commands also have optional arguments.
Instead of braces, these are passed within brackets \verb|[]|.
A basic example is the \cmd{sqrt} command
that produces not only square roots, but general roots:
\begin{VerbatimOut}{\jobname.tmp}
$\sqrt{7}$,
$\sqrt[3]{7}$.
\end{VerbatimOut}
\ShowExample

Another example is the optional argument to \cmd{cite} command,
which indicates a precise position within the cited work:
\verb|\cite[Page 40]{TLC}| produces \cite[Page 40]{TLC}.
(Here \verb|TLC| is the bibliography key for \emph{The \LaTeX{} Companion};
we will discuss this more in \Cref{sec:bibliography})

\begin{gotcha}
Inside an optional argument, \LaTeX{} interprets the first \verb|]| it sees as the closing bracket.
This means that if you need to use \verb|[]| \emph{inside} an optional argument
(say, as an optional argument to an inner command), you need to wrap things in braces.
Quite often, you might need even \emph{two sets of braces}.
Compare the two citations:
%
\begin{VerbatimOut}{\jobname.tmp}
\cite[{{Page [40] maybe?}}]{TLC}\\
\cite[Page [40] maybe?]{TLC}
\end{VerbatimOut}
\ShowExample
%
The first citation is shown correctly.
In the second, the optional argument is interpreted to be ``\verb|Page [40|''
and the citation name then ``\verb|m|'' (only a single letter since it was not wrapped in braces).
The remaining ``\verb|aybe?]{TLC}|'' is then normal body text.
\end{gotcha}

\begin{gotcha}
\LaTeX{} ignores whitespace between a command and its optional argument.
This might sometimes lead to surprising behaviour.
Here the second list item is interpreted as an optional argument that replaces the bullet symbol:
\indexcmd{item}
%
\begin{VerbatimOut}{\jobname.tmp}
\begin{itemize}
\item Cookies
\item [Maybe cake?]
\item Drinks
\end{itemize}
\end{VerbatimOut}
\ShowExample
\end{gotcha}


The syntax for optional arguments is not always clear:
sometimes they precede the real arguments, sometimes they follow them.
Sometimes they consist of just one thing,
sometimes they can be a list of things
(see the discussion of \cmd{usepackage} later in this chapter).

\todo{Defining commands with optional arguments?}


%
\subsection{Replacing existing commands}

Sometimes it is useful to overwrite an existing command.
One common example is rewriting \cmd{epsilon} to actually mean \cmd{varepsilon},
since many consider the variant $\varepsilon$ prettier than the regular $\epsilon$.

It is not possible to write \verb|\newcommand{\epsilon}{\varepsilon}|,
since \LaTeX{} rightly complains about \verb|\epsilon| being already defined.
You could be accidentally overwriting a command used elsewhere in the document.
You have to make your intentions clear by using \cmd{renewcommand}:

\begin{VerbatimOut}{\jobname.tmp}
Old epsilon: $\epsilon$,\\
\renewcommand{\epsilon}{\varepsilon}
New epsilon: $\epsilon$.
\end{VerbatimOut}
\ShowExample
%
Again, the redefinition of a command lasts only for the current scope.
Once the scope is closed, \verb|\epsilon| again means the old regular symbol.
To redefine a command within the entire document,
you need to call \cmd{renewcommand} in the document preamble.

Here is an example of \cmd{newcommand} and \cmd{renewcommand} with arguments.
The two commands have matching syntax.
Note that the new definition is free to have a different number of arguments.

\begin{VerbatimOut}{\jobname.tmp}
\newcommand{\dual}[2]{(#1 \mid #2)}
Mathematicians: $\dual{a}{b}$.\\
\renewcommand{\dual}[2]
  {\langle#2 \mid #1\rangle}
Physicists: $\dual{a}{b}$.
\end{VerbatimOut}
\ShowExample


\todo{Advanced: adding to an existing command.}


%
\subsection{Defining environments}

\index{environments}
Environments encapsulate larger blocks of text.
They also form implicit groups.

\begin{VerbatimOut}{\jobname.tmp}
\begin{center}
\renewcommand{\epsilon}{\varepsilon}
Centered text and $\epsilon$
\end{center}

\begin{flushright}
Right-aligned text.
Here we have the old $\epsilon$.
\end{flushright}
\end{VerbatimOut}
\ShowExample

To define an environment, we use the \cmd{newenvironment} command.
This command takes three arguments: the name of the environment
and code to expand at the beginning and the end of the environment.

\begin{VerbatimOut}{\jobname.tmp}
\newenvironment{cool}
    {A mathmo enters the lab.\par}
    {\par The mathmo leaves the lab.}

\begin{cool}
A massive explosion occurs.
\end{cool}
\end{VerbatimOut}
\ShowExample
%
(If you're wondering about the \cmd{par} commands,
they are used to ensure that the first and last lines are their own paragraphs.)

There is also \cmd{renewenvironment} that works like \verb|\renewcommand|.

Since environments are groups,
it is possible to change font characteristics for the duration of the environment:

\begin{VerbatimOut}{\jobname.tmp}
\newenvironment{mouse}{\tiny}{}

Ordinary text.
\begin{mouse}
Very small and very squeaky text.
\end{mouse}
Again ordinary text.
\end{VerbatimOut}
\ShowExample
%
Note that here the end code is empty;
the font properties are reset automatically as the group ends.

\todo{Parameter form; improvements in L3}


%
%
%
\section{Diagnosing errors}

Macro languages were popular in the 1980s, back when 640 kilobytes was enough memory for anyone.
An unfortunate consequence of \TeX's stability is that the compiler still operates under this worldview.
The source code is read exactly once from top to bottom,
and if the code is not correct, error messages can be extremely cryptic.

\begin{overleaf}
Some \LaTeX{} distributions are set up to ignore some errors.
For example, Overleaf is quite good at ignoring missing \verb|$| signs and other small typos.
You should really pay attention to red symbols in the source code margin
and next to the \emph{Recompile} button!
\end{overleaf}

\todo{Understanding error messages}


\section{Document structure}\label{sec:document structure}

\todo{Finding documentation for packages}


\section{Creating your own style file}


\section{Counters}



\chapter{Tools for serious publications}

\section{Cross-references and hyperlinks}\label{sec:crossref}

One of the main benefits of a programmable typesetting system is
that it takes care of numbering for you.
I can say ``We talk about floats in \Cref{sec:floats}, which starts on page~\pageref{sec:floats}'',
and be sure that the numbers are correct.
Moreover, the numbers are clickable hyperlinks in the PDF version.

The basic infrastructure consists of two parts.
First, the \cmd{label} command is used to annotate points that we want to refer to.
This command points to the latest numbered thing:
it could be a sectioning command, a numbered list item, a theorem environment,
or a numbered equation.

\begin{practices}
Put the \cmd{label} command immediately after the numbered thing.
This is both for readability and for maintainability.
\end{practices}

The \cmd{ref} command reproduces the number that the label was associated to.
There is also a \cmd{pageref} command that outputs the page number instead.
Finally, \cmd{eqref} puts the number inside parentheses just like in the equation tag.
%
\begin{VerbatimOut}{\jobname.tmp}
\begin{equation}\label{eq:pythagoras}
  a^2 + b^2 = c^2.
\end{equation}

As we see from Equation~\ref{eq:pythagoras},
that is \eqref{eq:pythagoras},
on page~\pageref{eq:pythagoras}\dots
\end{VerbatimOut}
\ShowExample
%
(Note the \verb|~| that keeps the prefix and number on the same line.
This is a good practice.)

\begin{practices}
The label can be quite free-form,
and I do not dare give any advice on how things should be named.
I personally use a quite verbose style with spaces,
like \texttt{eq:2d variational formula},
but you could also argue against that.

However, one nice and common practice is to prefix the label with a type.
Above, I have used \verb|eq:| to denote equations.
Other common ones are
\verb|def|(inition), \verb|fig|(ure), \verb|rem|(ark), \verb|sec|(tion),
\verb|tbl| (table), and \verb|thm| (theorem).
\end{practices}

In the writing phase, it is useful to see the keys in PDF output.
The \pkg{showkeys} package does just that:
wherever a \cmd[shown in margin]{label} is defined,
the key is printed to the page margin.
Check out the package documentation for options;
in particular, the \verb|notref| and \verb|notcite| options
suppress the display of keys at \verb|\ref| and \verb|\cite| commands respectively.



%
%
\subsection{Keeping them correct}\label{sec:cleveref}

So you go about writing, adding references to Theorem~4.1 or something similar.
Then near the end of your writing process,
you decide to call it a Lemma instead.
\LaTeX{} takes care of the numbering,
but what about all those references coded as \verb|Theorem~\ref{...}|?
Chances are, you forget to change some of them to \verb|Lemma~\ref{...}|,
and get referee feedback regarding non-existent results\dots%
\footnote{On a positive note: such a referee has really read the paper.}

Shouldn't \LaTeX{} also take care of the reference types?
Isn't that exactly the kind of problem computers are supposed to solve?
Enter the little package called \pkg{cleveref}.

\begin{practices}
If the preceding paragraph sounded like a true story that has happened to the present author,
you are absolutely right.

Almost all journal styles are compatible with \pkg{cleveref}.
Use it.
\end{practices}

This package adds two more commands: \cmd{cref} and \verb|\Cref|.
They do exactly what you would expect them to:
you write \verb|\Cref{thm:pythagoras}| or \verb|\cref{...}|,
and it outputs ``Theorem~4.1'' or respectively ``theorem~4.1''.
That is, the two commands differ on the casing of the first letter.

The command automatically supports all standard \LaTeX{} labels.
Theorem types are recognized if you define theorems with
\pkg[with cleveref]{amsthm} or \pkg[with cleveref]{ntheorem} (as you should).
See the package documentation for extension options.

A nice feature of this package is that you can also pass several references,
and \pkg{cleveref} sorts out how to present them.
That is,
\begin{ExampleCode}
\Cref{thm:first, thm:second, thm:third}
\end{ExampleCode}
might produce something like ``Theorems~4.1 to~4.3''
or ``Theorems~4.1, 4.7, and~5.1''.
Again, see the package documentation for how to customize this
(for example, if the plural form is not formed by just appending `s').

There are also the \cmd{cpageref} and \verb|\Cpageref|
commands to produce page references (with the same support for ranges),
and a few other more specialized commands.

\begin{gotcha}
Since \pkg[should be loaded last]{cleveref} modifies the referencing mechanism,
it should be loaded after any other packages that do so.
\end{gotcha}

\begin{gotcha}
The \pkg{cleveref} package has some support for non-English languages;
you can pass a language name as an option when you load the package.
However, label names are always output in the nominative case,
so the usefulness of the package may be limited for most other languages.
\end{gotcha}



%
%
%
\section{Bibliography management}\label{sec:bibliography}

There are three technologies to create bibliographies,
so let us go over them in the order of increasing preferability.
Finally, we also mention an unfortunate hack that some journals apparently require.

%
\subsection{Citation commands and the manual way}

Regardless of the method of creating a bibliography,
citations are inserted into the text with the \cmd{cite} command.
It takes one argument, the citation key.
Several citations can be combined by separating the keys with commas.
An optional argument can be used to specify a location within the citation.

\begin{VerbatimOut}{\jobname.tmp}
Documentation can be found in
\cite{TLC, tikz}.

See \cite[Chapter~8]{TLC}.
\end{VerbatimOut}
\ShowExample
(Note again the use of \verb|~| to keep the location on one line.)

\todo{Customizing citation styles}

The bibliography is then created within a \env{bibliography} environment.
This environment takes a single parameter,
which is the maximum length of a citation number.
Interestingly, it is given as a string where the length is used as the parameter.
That is, \verb|9| means that 1~character is reserved,
and \verb|99| means that 2~characters are reserved.

The bibliography is effectively a list environment
where entries are created by the \cmd{bibitem} command.
It takes the citation key as its sole parameter.
Any text following this command is the bibliography entry,
and you are free to format it as you like.

\begin{ExampleCode}
\begin{thebibliography}{9}

\bibitem{Oksendal}
    Bernt \O{}ksendal,
    \textit{Stochastic Differential Equations:
    An Introduction with Applications},
    Springer,
    6\textsuperscript{th} edition, 5\textsuperscript{th} corrected printing,
    2010.

\end{thebibliography}
\end{ExampleCode}

\begin{practices}
This method is only advisable for short throwaway notes
when collecting a \verb|.bib| file would be too much of an effort.
If you use a proper reference management application,
this is almost never the case.
\end{practices}

\begin{technote}
It might be useful to wrap the bibliography in a \env{flushleft} environment.
Bibliography entries are often hard to justify nicely,
so this might give a visually more pleasing appearance.
\end{technote}



%
\subsection{BibTeX: the automatic way}

Already in Lamport's \LaTeX2$\varepsilon$ book \cite{lamport} a better way is described.
The \prg{bibtex} compiler whose working we discussed on page~\pageref{bibtex process}
automates the process of bibliography creation.
In a nutshell:
\begin{itemize}
\item You collect bibliographic data into a special \verb|.bib| file.
    \index{bib file@.bib file}
\item You write citations as usual.
\item BibTeX takes care of formatting the bibliography in your preferred style,
    only including the references you have cited.
\end{itemize}
%
For this to work, you need to run \prg{bibtex} as part of the compilation.
(This only needs to be done when the bibliography is changed.)

\begin{practices}
You should also automate the process of collecting the bibliography.
Reference management applications can help you organize your references
and are able to download information from bibliographic databases.

Most reference managers support exporting in the \verb|.bib| format.
One good, free application is Zotero\footnotemark.

A small note as a Zotero user (as of April 2024):
unfortunately, the \verb|.bib| export escapes \LaTeX{} commands like \verb|$| into \verb|\$|,
and \verb|^| into even more horrendous \verb|\textasciicircum|.
You need to manually correct mathematical symbols in titles.
\end{practices}
\footnotetext{\url{www.zotero.org}}

The \verb|.bib| file format consists of entries that begin with a \verb|@type| declaration,
followed by \verb|{}| that contains the citation key
and further data in key-value format.
This is probably easiest to explain with examples,
so let us see some common entry types.

\todo{Capitalization}

\todo{bib file format}

\todo{What is the authoritative reference?}

\todo{Bibliography styles}

\todo{nocite}

\todo{Submission to arXiv}



%
\subsection{BibLaTeX: the next generation}
\todo{The differences to BibTeX: `von', doi, etc.}

\todo{Backreferences}



%
\subsection{Embedding the bibliography}
\todo{Embedding the bibliography for a published version}




%
%
%
\section{Tables of contents}

A table of contents is printed by the \cmd{tableofcontents} command.
If you have used the standard \LaTeX{} sectioning commands,
everything works as it should.
See the notes in \Cref{sec:sectioning}
for how to abbreviate the section titles in the table of contents.

\todo{Starred sectioning commands}

If you need to manually add something to the table of contents,
you can use \cmd{addcontentsline}.
It takes three arguments:
the name of the contents list (here \verb|toc|), the sectioning level, and the entry to add.
For example:
\begin{ExampleCode}
\addcontentsline{toc}{section}{A phantom section}
\end{ExampleCode}

By default, the bibliography does not appear in the table of contents.
The easiest way to fix this is to load the \pkg{tocbibind} package.

It is also possible to create lists of tables and figures.
These are produced by the \cmd{listoftables} and \cmd{listoffigures} commands.
To add entries manually, you need to pass \verb|lot| and \verb|lof|
respectively to the \cmd{addcontentsline} command.


%
\subsection{\emph{Indexing*}}

In the uncommon case that you need to prepare an index,
let us very briefly go over the basic syntax.
You need to put \cmd{makeindex} in the preamble,
and the index is printed with \cmd{printindex}.

As with bibliographies, an external program is used to compile the index.
This used to be done with \prg{makeindex},
but nowadays a better option is \prg{upmendex};
the latter is fully compatible with \prg{makeindex} syntax
but additionally supports Unicode.

Index entries are added with the \cmd{index} command.
At the simplest, it takes the text to be inserted to the index.
The index will point to the page where command is executed.
However, the syntax also supports subindex entries
and customizing the look of the entry.

A subindex entry like ``counters, list''
is created by putting \verb|!| between the index key and the subindex key.
If you need to control the placement of the key,
or want to apply styling,
you can put \verb|@| between the sorting key and the visual appearance.

See the examples below for how this is done.
\begin{ExampleCode}
\index{counters} % Basic index entry for 'counters'
\index{counters!list} % Entry for 'counters, list'
\index{tex file@\texttt{.tex} file} % Sorted with the letter 'T', not '.'
\end{ExampleCode}

As a final note, there is also the \cmd{see} command
that yield ``\emph{see} \dots'' entries.
These index entries do not feature a page number,
so the command can be placed anywhere.
As a real example from these notes:
\begin{ExampleCode}
\index{dots per inch!\see{DPI}}
\end{ExampleCode}


\begin{practices}
A good index is hard to make, but extremely useful for a reader.
There are even books on the art of indexing.

These notes are not to be looked as a model for good indexing;
most of the index is created automatically by the commands
that format command, environment, and package names.
The manually created part of the index has not received the love it would deserve.
\end{practices}



%
%
%
\section{Tools for the editing process}

%
\subsection{Line numbers}

Some journals require line numbers for the referee copy of a submitted work.
\todo{Also double spacing.}
This is usually achieved with the \pkg{lineno} package.

The syntax is very simple.
Load the package and specify \cmd{linenumbers} in the preamble or beginning of the document.
You can pause the numbering with \cmd{nolinenumbers} and resume it with \cmd{linenumbers} again.

You can specify the \cmd{modulolinenumbers} command to print a number only every fifth line;
the gap can be specified as an optional argument like \verb|\modulolinenumbers[2]|
(for a number on every other line).

By default, equation environments are not numbered.
Load the package with \verb|[mathlines]| optional argument to enable this.

\begin{gotcha}
Built-in support for \pkg[with lineno]{amsmath}
was only included in July~2022,
so you may need to update packages in your \LaTeX{} distribution.
\end{gotcha}

There is also the \verb|[pagewise]| package option
to reset the line numbers for each page, but be advised:
it might require two \LaTeX{} runs to be correct.

\begin{technote}
This package needs to do some deep magic,
since line numbers get fixed very late in the typesetting process.
\end{technote}


%
\subsection{latexdiff}

The \prg{latexdiff} program is an external script that is immensely useful in editing.
It compares two \verb|.tex| files and outputs a \verb|.tex| file
where any differences between the two are highlighted.

\begin{practices}
Many referees and journal editors really appreciate getting a latexdiff along with a revised version.
They are also useful to pass between coauthors,
and sometimes even for your own reference.

To make this work, you should preserve a copy of each submitted version for later diffing.
(This combines really well with version control software,
but that is a topic for a different course.)
\end{practices}

The \prg{latexdiff} program is a Perl script,
so you will need a Perl interpreter installed.
It is usually installed by default on Linux and macOS systems,
but for Windows you will need to install it.%
\footnote{There is also a ``portable'' distribution called Strawberry Perl,
if you are uncomfortable about installing things.}

On Linux and macOS the usage is as simple as
\begin{ExampleCode}
perl path/to/latexdiff.pl old.tex new.tex > diff.tex
\end{ExampleCode}
It could be that \prg{latexdiff} is even on your search path,
in which case it suffices to execute \verb|latexdiff|.
\todo{Check how this usually is}

On Windows, things are slightly more complicated.
Windows writes the output of the script (through the \verb|>| operator)
in the UTF-16 encoding, which is not compatible with UTF-8.
If your document contains Unicode characters,
they will be garbled and probably fail to compile.
To fix this, we need to tell Windows to produce output in UTF-8.%
\footnote{The reason for this mismatch is historical.
The Windows command-line interface was introduced with Windows~NT in 1993,
and UTF-8 saw the light at around the same time.
It took some more years for UTF-8 to become the standard choice.
The developers of Windows have been as serious as the \LaTeX{} developers about backwards compatibility,
so the default is stuck to be UTF-16.}

In the legacy Command Prompt, you should execute the following commands:
\begin{ExampleCode}
chcp 65001
perl path/to/latexdiff.pl old.tex new.tex > diff.tex
\end{ExampleCode}
Here the \verb|chcp| command stands for ``choose code page'',
and UTF-8 is numbered to be 65001.
(Code pages are synonymous for \LaTeX{} encodings like \verb|latin1|.)

In Windows PowerShell, you can enter the legacy Command Prompt by executing \verb|cmd|.
(The present author is not aware of a robust PowerShell-native method that would work across
all the versions in active use.)

The new \verb|.tex| file produced by \prg{latexdiff}
is a merged copy of the two input files,
with additions wrapped inside \verb|\DIFadd|
and deletions wrapped inside \verb|\DIFdel| commands.
By default, these produce coloured and underlined text.

\begin{gotcha}
Unfortunately, \prg[problems]{latexdiff} sometimes produces \LaTeX{} code that fails to compile.
For example, a subscript and the text preceding it might be separated by a \verb|\DIFadd| command.

You need to manually fix up these issues,
which makes for an exercise in \LaTeX{} debugging.
If you keep your lines reasonably short,
at least you get some assistance from the line numbers.

\todo{How to avoid the problems}
\end{gotcha}

\begin{gotcha}
Sometimes the differencing algorithm also produces non-obvious results:
If you delete some text and add new text in its place,
you might not get two neat ``deletion'' and ``insertion'' blocks.
Instead, the diff is a salad of deleted, inserted, and kept-intact pieces.

It is impossible to develop perfect heuristics for detecting the intended changes.
In principle, you could manually edit the diff,
but most likely you should just keep it as is.
One can always look at the two versions side by side;
what is important is that the diff shows that \emph{something big} has changed at that position.
\end{gotcha}

If your document is split into multiple \verb|.tex| files,
you only need to give the paths to the main files.
The algorithm is clever enough to parse \verb|\input| and \verb|\include| commands.

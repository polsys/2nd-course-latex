\chapter{Tools for serious publications}

\section{Cross-references and hyperlinks}\label{sec:crossref}

One of the main benefits of a programmable typesetting system is
that it takes care of numbering for you.
I can say ``We talk about floats in \Cref{sec:floats}, which starts on page~\pageref{sec:floats}'',
and be sure that the numbers are correct.
Moreover, the numbers are clickable hyperlinks in the PDF version.

The basic infrastructure consists of two parts.
First, the \cmd{label} command is used to annotate points that we want to refer to.
This command points to the latest numbered thing:
it could be a sectioning command, a numbered list item, a theorem environment,
or a numbered equation.

\begin{practices}
Put the \cmd{label} command immediately after the numbered thing.
This is both for readability and for maintainability.
\end{practices}

\begin{technote}
\LaTeX{} recognizes the ``latest numbered thing''
by the \cmd{refstepcounter} command mentioned on page~\pageref{rem:refstepcounter}.
\end{technote}

The \cmd{ref} command reproduces the number that the label was associated to.
There is also a \cmd{pageref} command that outputs the page number instead.
Finally, \cmd{eqref} puts the number inside parentheses just like in the equation tag.
%
\begin{VerbatimOut}{\jobname.tmp}
\begin{equation}\label{eq:pythagoras}
  a^2 + b^2 = c^2.
\end{equation}

As we see from Equation~\ref{eq:pythagoras},
that is \eqref{eq:pythagoras},
on page~\pageref{eq:pythagoras}\dots
\end{VerbatimOut}
\ShowExample
%
(Note the \verb|~| that keeps the prefix and number on the same line.
This is a good practice.)

\begin{practices}
The label can be quite free-form,
and I do not dare give any advice on how things should be named.
I personally use a quite verbose style with spaces,
like \texttt{eq:2d variational formula},
but you could also argue against that.

However, one nice and common practice is to prefix the label with a type.
Above, I have used \verb|eq:| to denote equations.
Other common ones are
\verb|def|(inition), \verb|fig|(ure), \verb|rem|(ark), \verb|sec|(tion),
\verb|tbl| (table), and \verb|thm| (theorem).
\end{practices}

In the writing phase, it is useful to see the keys in PDF output.
The \pkg{showkeys} package does just that:
wherever a \cmd[shown in margin]{label} is defined,
the key is printed to the page margin.
Check out the package documentation for options;
in particular, the \verb|notref| and \verb|notcite| options
suppress the display of keys at \verb|\ref| and \verb|\cite| commands respectively.



%
%
\subsection{Keeping them correct}\label{sec:cleveref}

\begin{warning}
Some good things do not last.

The first version of these notes advocated for the \obspkg{cleveref} package,
which provided a simple and powerful interface.
However, the package needed to patch things deep inside the cross-referencing mechanism.
The package has not seen an update since~2018,
whereas \LaTeX{} core developers have been revamping the internals heavily.

The November~2024 \LaTeX~kernel update broke theorem references,
which could still be worked around with \pkg[with cleveref]{thmtools}.
Then the June~2025 kernel update broke even further references.

My recommendation is to replace \obspkg{cleveref} with \pkg{zref-clever}
and the define workaround commands (see below).
\end{warning}

So you go about writing, adding references to Theorem~4.1 or something similar.
Then near the end of your writing process,
you decide to call it a Lemma instead.
\LaTeX{} takes care of the numbering,
but what about all those references coded as \verb|Theorem~\ref{...}|?
Chances are, you forget to change some of them to \verb|Lemma~\ref{...}|,
and get referee feedback regarding non-existent results\dots%
\footnote{On a positive note: such a referee has really read the paper.}

Shouldn't \LaTeX{} also take care of the reference types?
Isn't that exactly the kind of problem computers are supposed to solve?
Luckily, there are some packages that do just it.

\begin{practices}
If the preceding paragraph sounded like a true story that has happened to the present author,
you are absolutely right.
\end{practices}

One of them is \pkg{zref-clever}.
It is marked as experimental but seems to be quite stable and regularly updated.

This package adds a \cmd{zcref} command
that does exactly what you would expect it to:
\verb|\zcref{thm:pythagoras}|,
and it outputs ``Theorem~4.1''.
The command automatically supports all standard \LaTeX{} labels.
Theorem types are recognized if you define theorems with
\pkg[cross-references]{amsthm} or \pkg[cross-references]{thmtools}.

Reference style can be customized by quite a lot
with the \cmd{zcsetup} command.
For example, \verb|\zcsetup{cap}| would capitalize the references, which is usually desired.
These can also be passed as an optional argument to \cmd{zcref}.
See the package documentation for all options.

A nice feature of this package is that you can also pass several references,
and \pkg{zref-clever} sorts out how to present them.
That is,
\begin{ExampleCode}
\zcref{thm:first,thm:second,thm:third}
\end{ExampleCode}
might produce something like ``Theorems~4.1 to~4.3''
or ``Theorems~4.1, 4.7, and~5.1''.
Again, see the package documentation for how to customize this.

There is also the \cmd{zcpageref}
command to produce page references (with the same support for ranges).

The package does have predefined names for many common theorem environments,
but in some cases you need to define the name by yourself.
See the how-to section in the package documentation.

\begin{remark}
The \pkg{zref-clever} package has some support for non-English languages,
even those that have declension.
See the package documentation for more.
\end{remark}


\begin{practices}
If you have used \obspkg{cleveref}, you may be able to convert to \pkg{zref-clever} by something like
\begin{ExampleCode}
% Old
\usepackage[capitalize]{cleveref}

% New
\usepackage{zref-clever}
\zcsetup{cap} % etc.
\newcommand{\Cref}{\zcref}
\end{ExampleCode}
Make sure that your document still compiles with no warnings and that the references look correct.
You might need to do some adjustments to package setup.
\end{practices}



%
%
%
\section{Bibliography management}\label{sec:bibliography}

It seems that every journal has a slightly different preferred style for bibliographic references.
Some require full author names, some only initials
-- some put \emph{et al.} already after one author, some only after three.
Not to speak about numbered, author-year, and abbreviated citations.
Again, something for computers to do.

There are three standard technologies to create bibliographies,
so let us go over them in the order of increasing preferability.
Finally, we also mention an unfortunate hack that some journals apparently require.

%
\subsection{Citation commands and the manual way}\label{sec:bibliography manual}

Regardless of the method of creating a bibliography,
citations are inserted into the text with the \cmd{cite} command.
It takes one argument, the citation key.
Several citations can be combined by separating the keys with commas.
An optional argument can be used to specify a location within the citation.

\begin{VerbatimOut}{\jobname.tmp}
Documentation can be found in
\cite{TLC, tikz}.

See \cite[Chapter~8]{TLC}.
\end{VerbatimOut}
\ShowExample
(Note again the use of \verb|~| to keep the location on one line.)

The bibliography is then created within a \env{bibliography} environment.
This environment takes a single parameter,
which is the largest expected citation number.
That is, \verb|9| means that 1~character is reserved for indentation,
and \verb|99| means that 2~characters are reserved.

The bibliography is effectively a list environment
where entries are created by the \cmd{bibitem} command.
It takes the citation key as its sole parameter.
Any text following this command is the bibliography entry,
and you are free to format it as you like.

\begin{ExampleCode}
\begin{thebibliography}{9}

\bibitem{Oksendal}
    Bernt \O{}ksendal,
    \textit{Stochastic Differential Equations:
    An Introduction with Applications},
    Springer,
    6\textsuperscript{th} edition, 5\textsuperscript{th} corrected printing,
    2010.

\end{thebibliography}
\end{ExampleCode}

\begin{practices}
This method is only advisable for short throwaway notes
when collecting a \verb|.bib| file would be too much of an effort.
If you use a proper reference management application,
this is almost never the case.
\end{practices}

\begin{technote}
It might be useful to wrap the bibliography in a \env{flushleft} environment.
Bibliography entries are often hard to justify nicely,
so this might give a visually more pleasing appearance.
\end{technote}



%
\subsection{BibTeX: the automatic way}

Already in Lamport's \LaTeX2$\varepsilon$ book \cite{lamport} a better way is described.
The \prg{bibtex} compiler whose working we discussed on page~\pageref{bibtex process}
automates the process of bibliography creation.
In a nutshell:
\begin{itemize}
\item You collect bibliographic data into a special \verb|.bib| file.
    \index{bib file@.bib file}
\item You write citations as usual.
\item BibTeX takes care of formatting the bibliography in your preferred style,
    only including the references you have cited.
\end{itemize}
%
For this to work, you need to run \prg{bibtex} as part of the compilation.
This program creates a \verb|.bbl| file that contains the actual bibliography.
(This only needs to be done when the bibliography is changed.)

\begin{remark}
When you submit your work to arXiv, you need to send the compiled \verb|.bbl| file.
\index{bbl file@.bbl file!arXiv}
The arXiv processing system does not run the \prg[arXiv]{bibtex} compiler.
This means that you should run it manually,
verify that the resulting bibliography is up to date,
and then send both the \verb|.tex| and \verb|.bbl| files.
You should then check that the bibliography and citations
appear correctly in the version compiled by arXiv.
\end{remark}

\footnotetext{\url{www.zotero.org}}
\footnotetext{\url{retorque.re/zotero-better-bibtex/}}
\begin{practices}
You should also automate the process of collecting the bibliography.
Reference management applications can help you organize your references
and are able to download information from bibliographic databases.

Most reference managers support exporting in the \verb|.bib| format.
One good, free application is Zotero\footnotemark.

A small note as a Zotero user (as of April 2024):
unfortunately, the \verb|.bib| export escapes \LaTeX{} commands like \verb|$| into \verb|\$|,
and \verb|^| into even more horrendous \verb|\textasciicircum|.
You can install the \emph{Better BibTeX} plugin to Zotero to get
customizable, better-quality export.\footnotemark
\end{practices}

\begin{gotcha}
The \prg[Unicode]{bibtex} compiler is not Unicode-aware,
so you need to write non-English characters with \TeX{} special commands.
If an author name begins with a non-English character,
it may be sorted into an odd position.
(You can use the \verb|key| entry to provide an alternative sorting key.)
\end{gotcha}

The \verb|.bib| file format consists of entries that begin with a \verb|@type| declaration,
followed by \verb|{}| that contains the citation key
and further data in key-value format.
This is probably easiest to explain with examples,
so let us see some common entry types.

\begin{ExampleCode}
@article {OnPyth,
    author = {Firstname Lastname and Secondauthor, Also},
    title = {On {Pythagoras'} Theorem},
    journal = {Journal of Fake Mathematics},
    year = {2024},
    month = {jan},
    volume = {13},
    number = {1},
    pages = {653--675},
    note = {Redacted in Feb~2024.},
    doi = {10.1234/5678}
}
\end{ExampleCode}

Let us dissect this example a bit:
\begin{itemize}
\item BibTeX keywords are not case-sensitive,
    so you could just as well write \verb|@ARTICLE| or \verb|aUtHoR| (please don't).
\item This example is of type \verb|@article|, which means a published journal article.
\item It can be cited with \verb|\cite{OnPyth}|.
\item Values for entries can be wrapped in \verb|{...}| like here or in \verb|"..."|.
\item The two authors (F.~Lastname and A.~Secondauthor) are separated with \verb|and|.
    Author names can be written in three formats:
    \begin{itemize}
        \item ``First von Last'',
        \item ``von Last, First'',
        \item ``von Last, Jr, First''.
    \end{itemize}
    The bibliography style defines how the name will be printed.
\item The bibliography style also defines if \emph{Each Word in The Title Will Be Capitalized},
    or if \emph{The title should follow sentence case}.
    To protect words from automatic (de-)capitalization, wrap them inside \verb|{}|;
    here \emph{Pythagoras} is protected from becoming \emph{pythagoras}.
\item Publication month should be given as a three-letter abbreviation.
\item The page range should have the \verb|--| en dash as per good style!
\item The \verb|note| entry can be used for anything else that should be printed in the bibliography.
\item The \verb|doi| entry is not understood by original BibTeX styles,
    but newer styles (such as those provided by \pkg{natbib}) display them as clickable links.
    Similarly, there is an \verb|eid| entry type for electronic article IDs
    used by purely electronic journals.
\end{itemize}
%
Let us then see what a bibliographic entry for a book could look like.

\begin{ExampleCode}
@book {Pyth2,
    author = {von Lastname, Firstname},
    title = {Pythagoras Reloaded},
    year = {2024},
    publisher = {Fake Press},
    address = {Kumpula, Finland},
    edition = {Third},
    series = {Misunderstanding Historical Mathematics},
    number = {17}
}
\end{ExampleCode}

Now it is also necessary to specify the publisher.
The optional \verb|address| field can be used to give the city of the publisher.
The edition of the book should be specified in words for BibTeX
(but as a number or a complete string like ``Third edition'' for BibLaTeX\dots).
There are also keys for book series.

If you want to cite an article that appears as part of a book,
you can use the \verb|@incollection| type.
This is similar to \verb|@book|,
but now the name of the article is specified with \verb|title|
and the name of book with \verb|booktitle|.
You should include both the authors of the article
and editors of the book in the data.
There is also the \verb|@inproceedings| variant for conference proceedings.

There are also the self-explanatory \verb|@mastersthesis| and \verb|@phdthesis| types;
the precise type can be specified with a \verb|type| key.
There are also \verb|@techreport| for volumes in a report series,
and \verb|@manual| for technical documentation.
See \Cref{tbl:bib entries} for the typical entry keys in these types.

How to cite a preprint?
Unfortunately, BibTeX is quite a bit older than arXiv and friends,
so there is no built-in support for referencing online repositories.
The best bet is to use the \verb|@unpublished| entry type
and write something like ``arXiv preprint yyyy.xxxxx'' in the \verb|note| field.

\begin{practices}
As mentioned above, the built-in BibTeX styles have no support for DOI codes either.
Even if your current style does not show them,
you should still include \verb|doi| entries in article data, since
\begin{enumerate}
\item The style used by a journal might understand DOIs;
\item BibLaTeX (presented below) understands them very well;
\item They are very useful when you need to cross-check the data.
\end{enumerate}
\end{practices}

\future{When there is no natural author}

\begin{table}
\centering
\begin{tabular}{>{\small} p{2.6cm}
    |>{\raggedright\small\ttfamily} p{4cm}
    |>{\raggedright\arraybackslash\small\ttfamily} p{4cm}}
\textbf{Entry type} & \textrm{\textbf{Required}} & \textrm{\textbf{Optional}}\\
\hline
{\verb|@article|}
& {author, title, journal, year}
& {volume, number, pages, month, \textrm{and preferably also} doi, eid}\\
\hline
{\verb|@book|}
& {author \textrm{or} editor, title, publisher, year}
& {volume \textrm{or} number, series, address, edition, month}\\
\hline
{\verb|@incollection|}
& {author, title, booktitle, publisher, year}
& {pages, chapter, \textrm{and those in \texttt{@book}}}\\
\hline
{\verb|@inproceedings|}
& {author, title, booktitle, year}
& {pages, organization, \textrm{and those in \texttt{@book}}}\\
\hline
{\verb|@manual|}
& {title}
& {author, organization, address, edition, year, month}\\
\hline
{\verb|@mastersthesis|\newline\verb|@phdthesis|}
& {author, title, school, year}
& {type, address, month}\\
\hline
{\verb|@techreport|}
& {author, title, institution, year}
& {number, address, month}\\
\hline
{\verb|@unpublished|}
& {author, title, note}
& {year, month}\\
\hline
{\verb|@misc|}
& {\textrm{At least one of} author, title, howpublished, year, month, note}
& {}
\end{tabular}
\caption{Some common BibTeX entry types and recommended fields.
All entry types also support the \texttt{note} field.}
\label{tbl:bib entries}
\end{table}

To print the bibliography, you put a \cmd{bibliography} command at the expected position.
This command takes the name of the \verb|.bib| file as its argument.
The command must be accompanied with a \cmd{bibliographystyle} command;
see below for some common values.

If you want to make a reference appear in the bibliography even though it is not cited in the text,
you can use the \cmd{nocite} command.
Putting \verb|\nocite{key}| anywhere creates an invisible reference to the key.
You can also specify \verb|\nocite{*}| to include all entries from the \verb|.bib| file.

The magic of BibTeX is with the bibliography styles.
These customize the look of the final output.
The simplest style is \verb|plain|, produced with \verb|\bibliographystyle{plain}|
somewhere in the document.
This style has numeric labels and references sorted by author and year.
A basic bibliography would thus be produced by:
\begin{ExampleCode}
\bibliography{my-refs}
\bibliographystyle{plain}
\end{ExampleCode}

Another built-in style is \verb|alpha|,
which produces short labels from author names and year;
for example [Mit23].
Most journals have their own preferred style.

If you prefer author-year type citations (Mittelbach 2023),
the cleanest way is to load the package \pkg{natbib} with the \verb|authoryear| option.
This requires replacing the bibliography style with \verb|plainnat| or other compatible style.

This package also provides some new citation commands:
\cmd{citep} puts the full citation in parentheses, whereas \cmd{citet} wraps only the year;
the additional \cmd{citeauthor} and \cmd{citeyear} should be quite self-descriptive.
See the package documentation for more details.

\begin{practices}
Even if you don't use author-year citations,
the \verb|plainnat| style is useful in its support for DOIs and URLs.
\end{practices}

There is a wide variety of pre-built styles in different packages.
If you still cannot find a good enough style,
you can build your own.
The style language is quite complicated, but there is a nice \prg{custom-bib} program
that does the effort for you.
It is a command-line application (often included in the \LaTeX{} distribution)
that asks a series of questions and then outputs a \verb|.bst| file
that you can reference with \cmd{bibliographystyle}.



%
\subsection{BibLaTeX: the next generation}

As good as BibTeX is, it has its drawbacks:
no support for Unicode, Internet support only depending on the style,
and limited understanding of names outside the ``First (von) Last (Jr.)'' pattern.

Enter BibLaTeX, an extremely versatile replacement.
Its documentation \cite{biblatex} is quite extensive,
so let us just compare the essential differences to BibTeX.
The \verb|.bib| database format is unchanged,
but BibLaTeX adds a lot of new entry and key types.

\begin{gotcha}
BibLaTeX is quite backwards-compatible and BibTeX ignores unrecognized keys,
so it is quite easy to write a database compatible with both.

However, there are minor differences in e.g.\ how the \verb|edition| key is handled.
Check the bibliography carefully before submitting!
\end{gotcha}

Most importantly, the \verb|doi|, \verb|eid| (electronic article ID), and \verb|url| keys
understood by \pkg{natbib} are natively supported by BibLaTeX.
If you don't need BibTeX compatibility,
you can specify a preprint with the new syntax:
\begin{ExampleCode}
@unpublished{Pyth,
    title = {On {Pythagoras'} relativity theory},
    author = {Firstname Lastname},
    year = {2022},
    eprint = {2211.16111v4},
    eprinttype = {arxiv}
}
\end{ExampleCode}
%
Supported \verb|eprinttype| values include \verb|arxiv|, \verb|jstor|, \verb|pubmed|, and \verb|hdl|.
These all produce clickable hyperlinks.

The syntax to using BibLaTeX is quite different.
The package is loaded with \verb|\usepackage{biblatex}|.
Here it is possible to pass customization options like the style (see below).
Then the \verb|.bib| files are specified with \cmd{addbibresource} commands.
Finally, \cmd{printbibliography} outputs the bibliography in the specified place.

That is, a complete minimal example would be:
\begin{ExampleCode}
% In the preamble
\usepackage{biblatex}
\addbibresource{my-refs}

% ... in the document
\printbibliography
\end{ExampleCode}

Moreover, to compile the bibliography you need to use the \prg{biber} program.
If you cannot use \prg{biber}, it is possible to pass \verb|backend=bibtex|
as a package option to use legacy \prg[with BibLaTeX]{bibtex} instead.
Do note that Unicode support is then unavailable,
and some more advanced features might not work.

\begin{overleaf}
Overleaf takes care of calling the right bibliography compiler automatically.
\end{overleaf}

Bibliography styles are customized with package options.
(When converting a project to BibLaTeX,
you should also remove any references to \pkg[do not use with BibLaTeX]{natbib} and friends.)
To use the author-year style, you only need to pass \verb|style=authoryear|.
The \pkg[do not use with BibLaTeX]{natbib} commands like \cmd{citep}
are also implemented by BibLaTeX when \verb|natbib=true| is passed to the package.

There are quite a few built-in styles \cite[Section~3.3]{biblatex},
and even more can be found in extension packages.

Let us finally comment on one cool feature provided by BibLaTeX,
if you are not yet convinced to convert to it: back-references.
Just put \verb|backref=true| as a package option,
and the bibliography now contains references to pages where the corresponding entries have been cited.
This is surprisingly useful!
(See the bibliography of these notes for an example.)

\begin{practices}
Joking aside, unfortunately many journals are still to adopt BibLaTeX.
This means that one still also needs to work with legacy BibTeX as well.
\end{practices}


%
\subsection{Embedding the bibliography}

Some journals require you to submit only one \verb|.tex| file,
with the bibliography embedded in the text.
This is no reason to go to the manual entry;
instead, you can copy and paste the BibTeX output:

\begin{enumerate}
\addtocounter{enumi}{-1}
\item Do all your editing, and only do the following as the last step.
\item Compile the bibliography with \prg[embedding]{bibtex}.
\item Open the produced \verb|.bbl| file.
    Observe that it contains a \env[produced by BibTeX]{bibliography} environment!
\item Select all and copy.
\item Replace the \verb|\bibliography| command in your document
    with the pasted environment.
\end{enumerate}

If you use BibLaTeX, there are some good news and some bad news.
Good news: you can use BibLaTeX with all its benefits,
since the journal system is not in your way!
Bad news: the \verb|.bbl| file is not readily copy-pasteable.

There is a hack for doing this in the form of \pkg{biblatex2bibitem} package.
Check out the package page on CTAN for more information.




%
%
%
\section{Tables of contents}

A table of contents is printed by the \cmd{tableofcontents} command.
If you have used the standard \LaTeX{} sectioning commands,
everything works as it should.
See the notes in \Cref{sec:sectioning}
for how to abbreviate the section titles in the table of contents.

Starred sectioning commands like \cmd{section*} do not produce a number,
and they are not added to the table of contents by default.
If you need to manually add something to the table of contents,
you can use \cmd{addcontentsline}.
It takes three arguments:
the name of the contents list (here \verb|toc|), the sectioning level, and the entry to add.
For example:
\begin{ExampleCode}
\addcontentsline{toc}{section}{A phantom section}
\end{ExampleCode}

By default, the bibliography does not appear in the table of contents.
If you have made the bibliography manually or with \prg[to table of contents]{bibtex},
you can load the \pkg{tocbibind} package or use \cmd{addcontentsline}.
If you use \pkg[to table of contents]{biblatex},
you can add the \verb|[heading=bibintoc]| optional argument to \cmd{printbibliography}.

It is also possible to create lists of tables and figures.
These are produced by the \cmd{listoftables} and \cmd{listoffigures} commands.
To add entries manually, you need to pass \verb|lot| and \verb|lof|
respectively to the \cmd{addcontentsline} command.


%
\subsection{\emph{Indexing*}}\label{sec:index}

In the uncommon case that you need to prepare an index,
let us very briefly go over the basic syntax.
You need to put \cmd{makeindex} in the preamble,
and the index is printed with \cmd{printindex}.

As with bibliographies, an external program is used to compile the index.
This used to be done with \prg{makeindex},
but nowadays a better option is \prg{upmendex};
the latter is fully compatible with \prg{makeindex} syntax
but additionally supports Unicode.

Index entries are added with the \cmd{index} command.
At the simplest, it takes the text to be inserted to the index.
The index will point to the page where command is executed.
However, the syntax also supports subindex entries
and customizing the look of the entry.

A subindex entry like ``counters, list''
is created by putting \verb|!| between the index key and the subindex key.
If you need to control the placement of the key,
or want to apply styling,
you can put \verb|@| between the sorting key and the visual appearance.

See the examples below for how this is done.
\begin{ExampleCode}
\index{counters} % Basic index entry for 'counters'
\index{counters!list} % Entry for 'counters, list'
\index{tex file@\texttt{.tex} file} % Sorted with the letter 'T', not '.'
\end{ExampleCode}

As a final note, there is also the \cmd{see} command
that yield ``\emph{see} \dots'' entries.
These index entries do not feature a page number,
so the command can be placed anywhere.
As a real example from these notes:
\begin{ExampleCode}
\index{dots per inch!\see{DPI}}
\end{ExampleCode}


\begin{practices}
A good index is hard to make, but extremely useful for a reader.
There are even books on the art of indexing.

These notes are not to be looked as a model for good indexing;
most of the index is created automatically by the commands
that format command, environment, and package names.
The manually created part of the index has not received the love it would deserve.
\end{practices}



%
%
%
\section{Tools for the editing process}

%
\subsection{Line numbers}

Some journals require line numbers for the referee copy of a submitted work.
This is usually achieved with the \pkg{lineno} package.

The syntax is very simple.
Load the package and specify \cmd{linenumbers} in the preamble or beginning of the document.
You can pause the numbering with \cmd{nolinenumbers} and resume it with \cmd{linenumbers} again.

You can specify the \cmd{modulolinenumbers} command to print a number only every fifth line;
the gap can be specified as an optional argument like \verb|\modulolinenumbers[2]|
(for a number on every other line).

By default, equation environments are not numbered.
Load the package with \verb|[mathlines]| optional argument to enable this.

\begin{gotcha}
Built-in support for \pkg[with lineno]{amsmath}
was only included in July~2022,
so you may need to update packages in your \LaTeX{} distribution.
\end{gotcha}

There is also the \verb|[pagewise]| package option
to reset the line numbers for each page, but be advised:
it might require two \LaTeX{} runs to be correct.

\begin{technote}
This package needs to do some deep magic,
since line numbers get fixed very late in the typesetting process.
\end{technote}


%
\subsection{Line spacing}

Some journals also require some extra space between lines,
typically so-called 1.5~spacing.
The \LaTeX{} internal parameters are a bit complicated, so it is easiest to load the
\pkg{setspace} package and use its commands
\cmd{singlespacing}, \cmd{onehalfspacing}, and \cmd{doublespacing} (possibly in the preamble).


%
\subsection{latexdiff}

The \prg{latexdiff} program is an external script that is immensely useful in editing.
It compares two \verb|.tex| files and outputs a \verb|.tex| file
where any differences between the two are highlighted.

\begin{practices}
Many referees and journal editors really appreciate getting a latexdiff along with a revised version.
They are also useful to pass between coauthors,
and sometimes even for your own reference.

To make this work, you should preserve a copy of each submitted version for later diffing.
(This combines really well with version control software,
but that is a topic for a different course.)
\end{practices}

The \prg{latexdiff} program is a Perl script,
so you will need a Perl interpreter installed.
It is usually installed by default on Linux and macOS systems,
but for Windows you will need to install it.%
\footnote{There is also a ``portable'' distribution called Strawberry Perl,
if you are uncomfortable about installing things.}

On Linux and macOS the usage is as simple as
\begin{ExampleCode}
perl path/to/latexdiff.pl old.tex new.tex > diff.tex
\end{ExampleCode}
It could be that \prg{latexdiff} is even on your search path,
in which case it suffices to execute \verb|latexdiff|.
\future{Check how this usually is}

On Windows, things are slightly more complicated.
Windows writes the output of the script (through the \verb|>| operator)
in the UTF-16 encoding, which is not compatible with UTF-8.
If your document contains Unicode characters,
they will be garbled and probably fail to compile.
To fix this, we need to tell Windows to produce output in UTF-8.%
\footnote{The reason for this mismatch is historical.
The Windows command-line interface was introduced with Windows~NT in 1993,
and UTF-8 saw the light at around the same time.
It took some more years for UTF-8 to become the standard choice.
The developers of Windows have been as serious as the \LaTeX{} developers about backwards compatibility,
so the default is stuck to be UTF-16.}

In the legacy Command Prompt, you should execute the following commands:
\begin{ExampleCode}
chcp 65001
perl path/to/latexdiff.pl old.tex new.tex > diff.tex
\end{ExampleCode}
Here the \verb|chcp| command stands for ``choose code page'',
and UTF-8 is numbered to be 65001.
(Code pages are synonymous for \LaTeX{} encodings like \verb|latin1|.)

In Windows PowerShell, you can enter the legacy Command Prompt by executing \verb|cmd|.
(The present author is not aware of a robust PowerShell-native method that would work across
all the versions in active use.)

The new \verb|.tex| file produced by \prg{latexdiff}
is a merged copy of the two input files,
with additions wrapped inside \verb|\DIFadd|
and deletions wrapped inside \verb|\DIFdel| commands.
By default, these produce coloured and underlined text.

\begin{gotcha}
Unfortunately, \prg[problems]{latexdiff} sometimes produces \LaTeX{} code that fails to compile.
For example, a subscript and the text preceding it might be separated by a \verb|\DIFadd| command.

You need to manually fix up these issues,
which makes for an exercise in \LaTeX{} debugging.
If you keep your lines reasonably short,
at least you get some assistance from the line numbers.
\end{gotcha}

\begin{gotcha}
Sometimes the differencing algorithm also produces non-obvious results:
If you delete some text and add new text in its place,
you might not get two neat ``deletion'' and ``insertion'' blocks.
Instead, the diff is a salad of deleted, inserted, and kept-intact pieces.

It is impossible to develop perfect heuristics for detecting the intended changes.
In principle, you could manually edit the diff,
but most likely you should just keep it as is.
One can always look at the two versions side by side;
what is important is that the diff shows that \emph{something big} has changed at that position.
\end{gotcha}

If your document is split into multiple \verb|.tex| files,
you only need to give the paths to the main files.
The algorithm is clever enough to parse \verb|\input| and \verb|\include| commands.



%
%
%
\section{Embedding PDF files}

Including figures in PDF~format is discussed in \Cref{sec:pictures},
but what about embedding complete documents?
The most common use case is, of course, the article-based PhD~thesis.

The \pkg{pdfpages} package solves this problem.
It provides an \cmd{includepdf} command that extracts a page range from a PDF~document.
The extracted pages are laid out on new pages,
centered and scaled to fit.

The package documentation clearly explains the customization options,
but really the only necessary argument is \verb|pages|.
It takes a range of pages like \verb|{1,3,5}| or \verb|7-11|.
To include all pages, just write \verb|pages=-|.

\begin{ExampleCode}
% Preamble
\usepackage[final]{pdfpages}

\includepdf[pages=-]{article1.pdf}
\end{ExampleCode}
%
The \verb|final| package option ensures that pages are embedded
even if the rest of the document is in \verb|draft| mode.

\begin{latexthree}
Accessibility information in embedded PDF files is lost,
and the package might have some other compatibility issues with \cmd{DocumentMetadata} features
(as of May~2025).
\end{latexthree}

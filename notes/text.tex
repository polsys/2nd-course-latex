\chapter{Styling text}

%
%
%
\section{The not so minor characters}


%
%
%
\section{Style and size}

\subsection{Styles}
\LaTeX{} provides three independent attributes for modifying the appearance of text.
These are:

\begin{description}
\item[Family] The basic shape of the font.
    There are three families by default: serif, \textsf{sans serif}, and \texttt{typewriter}.
    In these notes, the sans serif font is used to highlight package names,
    and the typewriter font is used for \LaTeX{} code.

\item[Series] This is the combination of weight and width of the font.
    Generally, one only needs to care about the normal and \textbf{bold} series.
    Those into graphical design know that this is only the tip of the iceberg;
    see \todo{talk about this in the font customization section?}

\item[Shape] Usual choices include upright, \textit{italic}, and \textsc{small capitals}.
    Technically, small caps is not a shape variant but an independent style altogether,
    but we are simplifying things here.
\end{description}

These can be freely combined (although non-default fonts might not support all combinations):
%
\begin{VerbatimOut}{\jobname.tmp}
\textsf{\textit{italic sans serif}}\\
\textbf{\texttt{bold typewriter}}
\end{VerbatimOut}
\ShowExample

The following table shows the most common font commands.
These come in two forms: the first takes the text as its argument,
and the second applies the change to all the following text
(until end of scope; see p.~\pageref{ex:font scope}).

\begin{figure}[h]
\centering
% What happens below is TeX magic, and something you should only use in
% an advanced context like showing off inside the code of a LaTeX lecture note.
% To understand how this works, look up \csname in the index of Knuth's TeXbook.
\newcommand{\example}[3]{\cmd{#1}\texttt{\{...\}} &%
    \texttt{\{}\cmd{#2}\texttt{ ...\}} & \csname#1\endcsname{#3}}
\begin{tabular}{l|l|l}
Command form & Declarative form & Output\\
\hline
\example{textrm}{rmfamily}{Serif family (default)}\\
\example{textsf}{sffamily}{Sans serif}\\
\example{texttt}{ttfamily}{Typewriter}\\[1em]
\example{textbf}{bfseries}{Bold series}\\[1em]
\example{textit}{itshape}{Italic shape}\\
\example{textsc}{scshape}{Small capitals}
\end{tabular}
\end{figure}

There are two important points to remember:
First, you should keep the semantics and the styling separate.
For example, these notes define a \verb|\pkg| command to highlight package names;
there are no \verb|\textsf| commands sprinkled around the code.%
\footnote{Moreover, this command adds an entry to the index.
Thus the code is not only semantically meaningful but also benefits the reader.}

Another is that you should not use \verb|\textit| to emphasize text.
It is much better to use \cmd{emph}:
first, it is semantically meaningful, and second, it supports nesting:%
\begin{VerbatimOut}{\jobname.tmp}
Sometimes you need to
\emph{emphasize} a point.
\emph{And sometimes your emphasis
contains \emph{an even more important}
point to consider.}
\end{VerbatimOut}
\ShowExample
%
If you need a declarative form (for use in an environment), it is \cmd{em}.


\subsection{But what about underlining?}

\LaTeX{} does not provide a command to underline text.
Many typography enthusiasts consider underlining bad style,
an unfortunate byproduct of mechanical typewriters where you could not (easily) change the font.

If you still want to underline things, there is the \pkg{ulem} package.
By default, it rewrites \cmd[underline]{emph} to use (nested) underlining instead of italic,
so you might want to load it as \verb|\usepackage[normalem]{ulem}|,
which keeps \verb|\emph| as it usually is.

In addition to \cmd{uline}, this package also provides the \cmd{sout} command
that lets you strike out text.\index{strike-out text}
There are also several styles of underlines:
%
\begin{VerbatimOut}{\jobname.tmp}
\uline{Underlined}\\
\uuline{Doubly so}\\
\sout{Struck out}\\
\uwave{Waves}\\
\dashuline{Dashes}\\
\dotuline{Dots}
\end{VerbatimOut}
\ShowExample


\subsection{Size}

\subsection{Verbatim text}



%
%
%
\section{Colour}\label{sec:colour}


%
%
%
\section{Footnotes}

\todo{Gotcha with whitespace}


\section{List structures}



%
%
%
\section{Sectioning commands}

\begin{gotcha}
If your section title contains mathematical symbols
and you have \pkg[PDF navigation]{hyperref} loaded (as you do, right?),
you will probably get a warning.
This is because the PDF navigation bar format does not support mathematical symbols.

To solve this issue, you can use the \cmd{texorpdfstring} command.
It takes two arguments: the first is the one to show in the document,
and the second is the (non-mathematical) alternative to show in PDF navigation.
\begin{ExampleCode}
\section{The \texorpdfstring{$\phi^4$}{phi4} measure}
\end{ExampleCode}
This code displays ``The $\phi^4$ measure'' in the document
but ``The phi4 measure'' in the navigation bar.
\end{gotcha}

\todo{Front, main, back matter}

%
%
%
\section{Hyphenation, language, and overfull hboxes}\label{sec:overfull}

\todo{Font encodings}

\todo{Special characters}

\todo{Do we need to talk about boxes somewhere?}

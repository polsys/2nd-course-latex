\chapter{Graphics with TikZ}

There are several methods for producing line graphics in \LaTeX.
We cover here TikZ, which is quite portable across compilers and extremely versatile.
It is contained in the \pkg{tikz} package.

There are several extension libraries that are bundled with TikZ and loaded with the \cmd{usetikzlibrary} command.
Moreover, many other packages are built with TikZ.
One such package is \pkg{pgfplots}, which provides much more tools for plotting data.

TikZ has bit of a learning curve
and its documentation spreads over no less than 1321~pages (as of April~2024).
We are able to only cover the essentials here.
An unofficial web version of the manual can be accessed at \url{tikz.dev}.

\begin{technote}
TikZ is an acronym for \emph{TikZ ist kein Zeichenprogramm}
-- ``TikZ is not a drawing program''.
From this we can learn two things:
\begin{itemize}
\item Many \LaTeX{} core developers come from German-speaking Europe.
\item That as useful as TikZ is,
    for complex graphics you probably should use a proper, visual graphics editor.
\end{itemize}
Of course, the the acronym is not only recursive,
but also oxymoronic, since TikZ is a program (written in the \TeX{} language)
that outputs graphics\dots
\end{technote}

\begin{technote}
Internally, TikZ consists of three layers:
compiler-specific support code for primitive drawing operations,
a core layer called PGF,
and the (relatively) user-friendly frontend called TikZ.

Due to this, you can see references to PGF scattered in various places;
for example the CTAN page for \pkg{tikz} redirects to that of \pkg{pgf}.
In practice, the two are synonymous.
\end{technote}


%
%
\section{Coordinates, nodes, and paths}

Let us begin with a very simple example:
%
\begin{VerbatimOut}{\jobname.tmp}
\begin{tikzpicture}
  \draw[->] (0,0) -- (2,0);
  \draw[->] (0,0) -- (0,2);
  \draw[dotted] (0,0) -- ({sqrt(2)}, 1);
\end{tikzpicture}
\end{VerbatimOut}
\ShowExample
%
TikZ pictures are defined inside a \env{tikzpicture} environment.
(For small pictures, there is also a \cmd{tikz} command
that takes the contents of the environment as its single argument.)
The output appears inline in text, just as with \verb|\includegraphics| and friends.

Inside this environment, there are drawing commands.
Each of these commands \emph{ends with a semicolon}.
Forget the \verb|;| and you will get an error message.

The \cmd{draw} command is the basic workhorse of TikZ.
The \verb|--| operation draws a line between the two coordinates.
Let us first look at the syntax of coordinates before going back to the optional arguments
and different operations.

There are several coordinate systems, of which I believe the next four are most common.
\begin{description}
\item[xyz] As in the example above, coordinates in this system are
    tuples or triples of numbers separated by commas.
    These are with respect to unit vectors defined in the environment options.
    By default, the $x$ and $y$ unit vectors are 1~cm long and along the usual axes.
    The $z$ axis is skewed towards the reader.

    It is possible to use mathematical syntax like \verb|+|, \verb|/|, and \verb|sqrt(...)|
    to give more complicated coordinate expressions.
    You might need to wrap the calculation inside \verb|{}|.

\item[canvas] The numbers can also have units like \verb|(1cm, 20pt)|.
    In this case the coordinates are the absolute length away from canvas origin.
    It is possible to use expressions like \verb|1cm + 2pt|.
    
    These units are still not completely absolute:
    the canvas can still be scaled with an optional argument.

    \emph{Warning}: While it is technically possible to mix canvas and xyz coordinates
    inside the tuple, it is ill-advised.
    Similarly, an expression like \verb|1+2cm| is interpreted as \verb|1pt+2cm|,
    not as ``unit vector plus 1~cm to the right''.

\item[xyz polar] Polar coordinates are like \verb|(30:2)|,
    which means: 30 degrees counterclockwise, radius 2~units.

    If the $x$ and $y$ unit vectors are set to non-default values,
    the angle corresponds to a point on the ellipse specified by $x$ and $y$ vectors,
    and the resulting vector is multiplied by the radius factor.

\item[canvas polar] Here the radius has a specified unit: \verb|(30:2cm)|.
\end{description}

There are also the barycentric coordinate system (weighted average of basis vectors)
and a possibility to compute tangents of shapes.
These are covered in \cite[Section~13.2]{tikz}.

Coordinates can be given names with the \cmd{coordinate} command.
In the example below, we also use the \verb|cycle| specifier to close the loop.
%
\begin{VerbatimOut}{\jobname.tmp}
\begin{tikzpicture}
  \coordinate (A) at (0:3cm);
  \coordinate (B) at (120:2cm);

  \draw (0,0) -- (A) -- (B) -- cycle;
\end{tikzpicture}
\end{VerbatimOut}
\ShowExample

It is also possible to specify relative coordinates:
by prefixing the coordinate with \verb|++|, it is relative to the previous position.%
\footnote{It is also possible to prefix with just \texttt{+},
in which case the current position is not updated.}
%
\begin{VerbatimOut}{\jobname.tmp}
\begin{tikzpicture}
  \draw (0,0) -- ++(0:3) -- ++(90:2)
    -- ++(180:1) -- ++(270:1/2);
\end{tikzpicture}
\end{VerbatimOut}
\ShowExample


%
%
\subsection{Nodes}

Nodes can be thought of as text placed in the picture.
The text can contain mathematics and formatting commands, and even pictures.

A node can be placed either as part of a path, or separately with a \cmd{node} command.
In the latter case, it is possible to refer to the node later.
%
\begin{VerbatimOut}{\jobname.tmp}
\begin{tikzpicture}
  \draw (0,0) -- (2,0) node {Left};
  \node (A) at (0,2) {Above};
  \draw (0,0) -- (A);
  \node at (0,-1) {Below};
\end{tikzpicture}
\end{VerbatimOut}
\ShowExample

Note the difference between the lines connecting the origin to the Left and Above nodes.
When a node is created as part of a path operation,
it is \emph{centered} at the specified coordinate.
That is, the center of the ``Left'' text is at $(2,0)$.
The position of the node can be customized by passing
\verb|above|, \verb|below|, \verb|left|, or \verb|right| as an optional argument.

On the other hand, when a path is drawn to a previously defined node,
TikZ tries to stop at the boundary of the node.
It is possible to specify the position on the border by cardinal directions (like \verb|north|)
or an angle; or to specify \verb|center| if you really mean the center.
These are given by the \verb|name.anchor| syntax as in the example below.

To make the boundary of the ``Above'' node visible,
we also pass the \verb|draw| option to the \cmd{node} command.
%
\begin{VerbatimOut}{\jobname.tmp}
\begin{tikzpicture}
  \draw (0,0) -- (2,0) node[right] {Left};
  \node[draw] (A) at (0,2) {Above};
  \draw (0,0) -- (A.west);
\end{tikzpicture}
\end{VerbatimOut}
\ShowExample

One more interesting positioning specifier is the optional \verb|pos| argument.
It is understood as a position (in the range 0 to 1) on the previous path segment.
It can be combined with other specifiers as in here:
%
\begin{VerbatimOut}{\jobname.tmp}
\begin{tikzpicture}
  \draw[->] (0,0) -- (3,0)
    node[pos=0.5,above] {$\frac 1 2$}
    node[pos=0.25,below] {$\frac 1 4$}
    node[pos=0.75,below] {$\frac 3 4$};
\end{tikzpicture}
\end{VerbatimOut}
\ShowExample

To embed pictures,
you can use \cmd[in TikZ]{includegraphics} as usual inside a node.
To illustrate this once more with our fur-shedding assistants:
%
\begin{VerbatimOut}{\jobname.tmp}
\centering
\newcommand{\dogfile}{pictures/TheDogs.jpg}
\begin{tikzpicture}
  \node[draw] (Both) at (0,0)
    {\includegraphics[height=3cm]{\dogfile}};
  \node[draw] (Netta) at (-5,0)
    {\includegraphics[bb=1cm 2cm 3cm 5cm, clip, height=2cm]{\dogfile}};
  \node[draw] (Cira) at (5,0)
    {\includegraphics[bb=2.8cm 1.4cm 7cm 6cm, clip, height=2cm]{\dogfile}};
  \draw[->] (Both.west) -- (Netta.east) node[pos=0.5, above] {Netta};
  \draw[->] (Both.east) -- (Cira.west) node[pos=0.5, above] {Cira};
\end{tikzpicture}
\end{VerbatimOut}
\ShowExampleBelow[2]


%
%
\subsection{Path options}

There are lots of options that can be passed to the drawing commands.
As the very first one, let us consider \verb|color|.
It takes a colour name as its argument;
\pkg[with TikZ]{xcolor} extensions and custom colours are supported.
It is possible to blend the color with white with the \verb|!| specifier
(see page~\pageref{ex:color blending}).
%
\begin{VerbatimOut}{\jobname.tmp}
\definecolor{Sciency}{cmyk}{0,0.46,1,0}
\begin{tikzpicture}
  \draw[color=blue] (0,0)--++(3,0);
  \draw[color=LimeGreen] (0,-0.5)--++(3,0);
  \draw[color=Sciency] (0,-1)--++(3,0);
  \draw[color=Sciency!60] (0,-1.5)--++(3,0);
\end{tikzpicture}
\end{VerbatimOut}
\ShowExample

There is also a range of line weights.
Do note that their names may include spaces.
Of these, \verb|thin| is the default.
%
\begin{VerbatimOut}{\jobname.tmp}
\begin{tikzpicture}[yscale=0.8]
  \draw[very thin] (0,-0.5)--++(3,0);
  \draw[thin] (0,-1)--++(3,0);
  \draw[semithick] (0,-1.5)--++(3,0);
  \draw[thick] (0,-2)--++(3,0);
  \draw[very thick] (0,-2.5)--++(3,0);
  \draw[ultra thick] (0,-3)--++(3,0);
\end{tikzpicture}
\end{VerbatimOut}
\ShowExample

Then, there are the line styles.
The basic ones are (it is possible to specify custom ones, but that you have to read from the docs):
%
\begin{VerbatimOut}{\jobname.tmp}
\begin{tikzpicture}[yscale=0.8]
  \draw[solid] (0,0.5)--++(3,0);
  \draw[dashed] (0,-0)--++(3,0);
  \draw[dotted] (0,-0.5)--++(3,0);
  \draw[densely dashed] (0,-1)--++(3,0);
  \draw[densely dotted] (0,-1.5)--++(3,0);
  \draw[loosely dashed] (0,-2)--++(3,0);
  \draw[loosely dotted] (0,-2.5)--++(3,0);
  \draw[dash dot] (0,-3)--++(3,0);
  \draw[dash dot dot] (0,-3.5)--++(3,0);
\end{tikzpicture}
\end{VerbatimOut}
\ShowExample

\begin{practices}
As remarked in the section on colour,
you should never use colour as the sole means of giving information.
In simple line graphics, the combination of colour and line style is often useful.
However, complex line styles are also hard to read!
\end{practices}


Finally, there are the arrows.
The syntax for arrow tips is $\langle\textit{start}\rangle$\verb|-|$\langle\textit{end}\rangle$,
where the start/end specification can contain one or more \verb|<|, \verb|>|, or \verb$|$.
%
\begin{VerbatimOut}{\jobname.tmp}
\begin{tikzpicture}[yscale=0.8]
  \draw[<->] (0,-0)--++(3,0);
  \draw[->>] (0,-0.5)--++(3,0);
  \draw[|->|] (0,-1.0)--++(3,0);
\end{tikzpicture}
\end{VerbatimOut}
\ShowExample
%
The \verb|arrows.meta| extension library contains many more tip styles
and options to customize their size; see \cite[Section~16]{tikz}.
%
\begin{VerbatimOut}{\jobname.tmp}
% \usetikzlibrary{arrows.meta}
\begin{tikzpicture}[yscale=0.8]
  \draw[Circle-{Circle[open]}] (0,-0)--++(3,0);
  \draw[-{Stealth[length=5mm,width=2mm,red]}] (4,0)--++(3,0);
  \draw[{->[harpoon]}] (8,0)--++(3,0);
\end{tikzpicture}
\end{VerbatimOut}
\ShowExampleBelow

One more option: the \verb|rounded corners| option does what is says.
It takes as its argument a radius of the corner.
It is one of the few options that can be changed in the middle of a path
(none of the above can, unfortunately).
%
\begin{VerbatimOut}{\jobname.tmp}
\begin{tikzpicture}
  \draw[rounded corners=2mm] (0,0) -- (2,0)
    [rounded corners=5mm] -- (2,-2)
    [sharp corners] -- (0,-2) -- (0,-1);
\end{tikzpicture}
\end{VerbatimOut}
\ShowExample


%
%
\subsection{\emph{Absolute positioning*}}\label{sec:tikz absolute}

In some very special cases it might be useful to position graphics at absolute coordinates on the page.
This can be done by specifying coordinates relative to the \verb|current page| anchor,
and specifying two optional arguments to the environment:
\begin{description}
\item[\texttt{remember picture}] tells TikZ to remember the position on the page between \LaTeX{} runs.
    You might need to compile twice.
\item[\texttt{overlay}] tells TikZ to ignore bounding box computations.
    This ensures that the picture does not interact with the rest of the page contents.
\end{description}

For example, here we start from the right edge of the page,
move a bit upwards and to the left, and then fill a small circle:
%
\begin{VerbatimOut}{\jobname.tmp}
\begin{tikzpicture}[remember picture, overlay]
  \fill[blue] (current page.east) ++(-1cm, 2cm) circle [radius=0.2cm];
\end{tikzpicture}
\end{VerbatimOut}
\ExecuteExample
%
See \cite[Section~17.3.2]{tikz} for the full details.



%
%
%
\section{Special paths}

In addition to the \verb|--| style path joining two points,
there are the \verb$|-$ and \verb$-|$ styles.
These split the line into vertical and horizontal segments,
the order of which you can probably guess.
%
\begin{VerbatimOut}{\jobname.tmp}
\begin{tikzpicture}
  \draw (0,0) -| (2,1);
\end{tikzpicture}
\end{VerbatimOut}
\ShowExample

To draw a curved path between two points, you can use the \verb|to| operation
with optional arguments.
The \verb|in| and \verb|out| arguments take an angle in degrees (clockwise).
There are also the \verb|bend left| and \verb|bend right| options
for automatic control.
%
\begin{VerbatimOut}{\jobname.tmp}
\centering
\begin{tikzpicture}
  \draw (0,0) to[out=0,in=180] (2,1);
  \draw (3,0) to[bend left] (5,1);
  \draw[red,dashed] (3,0) to[bend right] (5,1);
\end{tikzpicture}
\end{VerbatimOut}
\ShowExampleBelow[2]

To draw rectangles, there is the \verb|rectangle| shorthand
that is put between coordinates of the two opposite corners.
Similarly, there is the \verb|circle| command that is put after the centre coordinate
and takes the radius as an optional parameter.
It is possible to specify \verb|x radius| and \verb|y radius| separately to get an ellipse
(which can be further rotated with the \verb|rotate| parameter).

Moreover, the command \cmd{fill} does exactly what you would expect it to;
it fills the enclosed region.
The combination \cmd{filldraw} both fills and draws.
If you have defined the path through line segments,
the path is automatically closed.%
\footnote{If the lines cross, the definition of interior can be customized with optional arguments
\texttt{nonzero rule} and \texttt{even odd rule}.
These are defined in \cite[Section~15.5.2]{tikz}.}
%
\begin{VerbatimOut}{\jobname.tmp}
\centering
\begin{tikzpicture}
  \draw (0,0) rectangle (3,2);
  \draw[red] (3/2,1) circle
    [x radius=3/2, y radius=1];
  \filldraw[blue, fill=blue!20] (4,1) circle [radius=0.5];
\end{tikzpicture}
\end{VerbatimOut}
\ShowExampleBelow[2]

There are also commands for drawing arcs, parabolas, and even Bézier curves;
see \cite[Section~14]{tikz}.

There is a helper command for drawing coordinate grids.
TikZ provides a shorthand style \verb|help lines|
that can be customized as in \Cref{sec:tikz styles}.
Here we overlay two grids with different step sizes (default is one unit).
%
\begin{VerbatimOut}{\jobname.tmp}
\centering
\begin{tikzpicture}
  \draw[help lines, xstep=0.5, ystep=0.5] (0,0) grid (4,3);
  \draw (0,0) grid (4,3);
  \draw[red,very thick,->] (0,0) -- (2,1);
\end{tikzpicture}
\end{VerbatimOut}
\ShowExampleBelow[2]

Now that we know how to draw a coordinate grid,
it is very natural to draw some functions on it!
TikZ includes a reasonably complex calculation engine,
which makes it possible to draw some complicated functions.
%
\begin{VerbatimOut}{\jobname.tmp}
\centering
\begin{tikzpicture}
  \draw[help lines] (0,0) grid (8,3);
  \draw[red, domain=0:8, samples=100] plot (\x, {1+sin(\x r)})
    node[right] {$1+\sin(x)$};
  \draw[blue, dashed, domain=0:8, samples=100] plot (\x, {exp(\x/8)})
    node[right] {$\exp(x/8)$};
  \draw[black, domain=0:8, samples=100] plot (\x, {pow(\x, 1/2)})
    node[above right] {$\sqrt x$};
\end{tikzpicture}
\end{VerbatimOut}
\ShowExampleBelow[2]
Some notes on the usage:
\begin{itemize}
\item The domain is specified using the syntax \verb|lower:upper|.
\item The number of samples is optional to specify, but the default value is quite small.
\item The variable $x$ is written with a backslash \verb|\x|,
    but the mathematical commands are spelled without a backslash.
\item Mathematical expressions should be wrapped inside \verb|{}|.
\item Powers like $x^2$ can be specified with \verb|pow(\x, 2)|.
\item Trigonometric functions assume degrees by default;
    the suffix \verb|r| after \verb|\x| converts the variable into radians.
\end{itemize}

Note that the plot is still defined as a pair of $(x,y)$ coordinates.
This makes it possible to draw parametric plots.
The name of the variable can be customized.
%
\begin{VerbatimOut}{\jobname.tmp}
\centering
\begin{tikzpicture}
  \draw[help lines] (-2,-1) grid (2,2);
  \draw[red, domain=1:15, samples=150, variable=\t]
    plot ({2*cos(\t r) / \t}, {2*sin(\t r) / \t});
\end{tikzpicture}
\end{VerbatimOut}
\ShowExampleBelow[2]

It is also possible to plot individual points or bar graphs,
or to even load a dataset from an external file.
See \cite[Section~22]{tikz} for more,
but also note the warnings below.

\begin{warning}
``Division by zero'' and ``number too big`` errors
can sometimes lead to confusing (and repeated!) error messages.
\end{warning}

\begin{practices}
TikZ is excellent for producing small and simple function plots
in a style consistent with the rest of the document.
However, compilation times can get quite long for even slightly complicated functions.

If you need to plot lots of complex functions,
you should precompute the function values in a file
and use the data plotting facility of TikZ; see \cite[Section~22]{tikz}.
\end{practices}



%
%
\section{Transformations}

The \verb|xshift|, \verb|yshift|, \verb|rotate|, and \verb|scale| optional arguments
can be used to locally change the coordinate system.%
\footnote{There are some more; see \cite[Section~25.3]{tikz} for details.
A particular example is \texttt{rotate around}, which rotates around a specified point.}
You can use a \env{scope} environment within the picture
to pass the transformation to multiple drawing operations:
%
\begin{VerbatimOut}{\jobname.tmp}
\centering
\begin{tikzpicture}
  \draw (0,0) rectangle (2,1);
  \begin{scope}[xshift=3cm]
    \draw[dotted] (0,0) rectangle (2,1);
  \end{scope}
  \begin{scope}[xshift=6.2cm, scale=0.5, rotate=30]
    \draw[blue] (0,0) rectangle (2,1);
  \end{scope}
\end{tikzpicture}
\end{VerbatimOut}
\ShowExampleBelow[2]
%
Note that the \verb|xshift| parameter should be specified with a unit
(otherwise the unit is assumed to be \verb|pt|).
The \env{scope} environment can be used to pass any optional arguments to enclosed commands;
for example, color or line style declarations can be passed as well.

Sometimes it is necessary to extend or to shrink the bounding box of the picture.
For this, the \cmd{useasboundingbox} command is useful.%
\footnote{Let us note that this is a shorthand for
\texttt{\textbackslash{}path[use as bounding box]};
TikZ certainly has many ways to do common operations.}
(This is especially useful in a Beamer slide where you gradually reveal parts of the picture.
You can use this command to reserve enough space for the full picture.)
\label{ex:tikz useasboundingbox}

Here we instead shrink the bounding box
so that the picture contents extend beyond the reserved space.
The dramatic effect is somewhat ruined by displaying the bounding box.
%
\begin{VerbatimOut}{\jobname.tmp}
\centering
Then a huge seagull
\begin{tikzpicture}
  \useasboundingbox[draw,dotted,gray] (-1em,-0.7em) rectangle (1em,0.8em);

  \draw (0,0) circle [radius=0.7em];
  \fill[orange] (-0.3em,-0.1em)--(0.3em,-0.1em)--(0em,-1em)--cycle;
  \draw (0.7em,0em) parabola bend (5em,1em) (6em,0.8em);
  \draw (-0.7em,0em) parabola bend (-5em,1em) (-6em,0.8em);
\end{tikzpicture}
came swooping down.
\end{VerbatimOut}
\ShowExampleBelow[2]



%
%
\section{Loops}\label{sec:pgffor}

TikZ automatically loads the \pkg{pgffor} package that provides a useful \cmd{foreach} command.
It can be used in two ways.
The first is to loop over a given set:
%
\begin{VerbatimOut}{\jobname.tmp}
\centering
\begin{tikzpicture}
  \foreach \x in {0,1,2,3}
    \draw (\x, 0) circle [radius=0.5] node {$\x$};
\end{tikzpicture}
\end{VerbatimOut}
\ShowExampleBelow[2]
%
Loops can be nested:
%
\begin{VerbatimOut}{\jobname.tmp}
\centering
\begin{tikzpicture}
  \foreach \x in {1,2,3}
    \foreach \y in {1,2,3}
      \draw (\x, -\y) circle [radius={(\x+\y)/15}];
\end{tikzpicture}
\end{VerbatimOut}
\ShowExampleBelow[2]
With \verb|{}|, it is possible to define several commands to be executed.
It is also possible to define several variables in one loop:
%
\begin{VerbatimOut}{\jobname.tmp}
\centering
\begin{tikzpicture}
  \foreach \x/\col in {1/red, 3/black, 5/blue} {
    \draw[\col] (\x, 0) circle [radius=0.5] node {$\x$};
    \draw[->] (\x,-0.2) -- (\x,-1) node[below] {\col};
  }
\end{tikzpicture}
\end{VerbatimOut}
\ShowExampleBelow[2]

Another way to use \cmd{foreach} is to specify a range.
It takes the first, second, and last element, and \verb|...| before the final element.
%
\begin{VerbatimOut}{\jobname.tmp}
\centering
\begin{tikzpicture}
  \foreach \x in {0,1,...,5}
    \draw (\x, 0) circle [radius=0.5] node {$\x$};
\end{tikzpicture}
\end{VerbatimOut}
\ShowExampleBelow[2]

Finally, let us note that the \cmd{foreach} command can freely be used outside TikZ.
(If you do not need full TikZ, you can just load the \pkg{pgffor} package.)
%
\begin{VerbatimOut}{\jobname.tmp}
\foreach \dog/\yr in {Cira/2009, Netta/2016}
  {\dog{} was born in \yr{}.\\}
\end{VerbatimOut}
\ShowExample


%
%
\section{\emph{Setting styles*}}\label{sec:tikz styles}

In the grid example, we saw the \verb|help lines| style that produced thin, gray lines.
TikZ provides a general facility for setting up such shorthand styles.
If you find yourself repeating graphical options,
you should consider using a style instead.%
\footnote{Those familiar with web development notice that this system is similar to
Cascading Style Sheets (CSS); indeed, the TikZ style system is inspired by it.}

To customize the style,
you use the \verb|/.style={}| syntax in the optional argument to the picture environment:
%
\begin{VerbatimOut}{\jobname.tmp}
\centering
\begin{tikzpicture}[help lines/.style={very thick,blue}]
  \draw[help lines] (0,0) grid (3,2);
\end{tikzpicture}
\end{VerbatimOut}
\ShowExampleBelow[2]

If you only want to customize part of the style,
you can keep the previous elements in place by using \verb|/.append style={}| instead.
Here we keep the lines thin and gray, but also make them dashed:
%
\begin{VerbatimOut}{\jobname.tmp}
\centering
\begin{tikzpicture}[help lines/.append style={dashed}]
  \draw[help lines] (0,0) grid (3,2);
\end{tikzpicture}
\end{VerbatimOut}
\ShowExampleBelow[2]

You can define your own styles with the \verb|/.style| syntax.
To reuse them across multiple TikZ environments,
you can put the definition in a \cmd{tikzset} command.
Such a command could go into the preamble.
It is still possible to override styles locally.
Here we define a style called \verb|highlight|:
%
\begin{VerbatimOut}{\jobname.tmp}
\centering
\tikzset{highlight/.style={draw,very thick,orange}}
\newcommand{\dogfile}{pictures/TheDogs.jpg}

\begin{tikzpicture}
  \node[highlight] at (-2,0)
    {\includegraphics[bb=1cm 2cm 3cm 5cm, clip, height=2cm]{\dogfile}};
  \node[highlight, blue] at (2,0)
    {\includegraphics[bb=2.8cm 1.4cm 7cm 6cm, clip, height=2cm]{\dogfile}};
\end{tikzpicture}
\end{VerbatimOut}
\ShowExampleBelow[2]


\begin{practices}
Do note that this idea goes very well with the separation of content and presentation.
If your document contains a lot of TikZ graphics with repeated graphical styles,
you should use styles with descriptive names.
Not only is the code semantically meaningful, but also easier to change.
\end{practices}


%
%
\section{Some extensions}

TikZ comes with several extension libraries that are not loaded by default.
They can be enabled with \cmd{usetikzlibrary}.
Their full documentation fills \cite[Part~V]{tikz},
but let us briefly show a couple illustrative examples.

\begin{practices}
TikZ has a lot of features,
\emph{aber es ist kein Zeichenprogramm}.
Some of the features are total overkill to do as part of \TeX{} compilation.
Also remember that simplicity is often better in scientific illustration!
\end{practices}


%
\subsection{Pattern fills}

The \verb|patterns| extension library allows filling shapes with a pattern.
In some cases it might be a visually appealing alternative to colour.

\begin{VerbatimOut}{\jobname.tmp}
\centering
% In the preamble:
% \usetikzlibrary{patterns}

\begin{tikzpicture}
\fill[pattern=north east lines] (-1,-1) rectangle (1,1);
\fill[pattern=dots, draw] (3,0) circle[radius=1];
\end{tikzpicture}
\end{VerbatimOut}
\ShowExampleBelow[2]


%
\subsection{Path decorations}

If straight and curved lines are not enough for you,
it is possible to add some visual extravaganza.
There are a couple basic types of decorations
(and quite a few less basic ones -- see the documentation if you dare),
separated into their own libraries.
A useful overview is at \cite[Section~24]{tikz}.

First, the decoration can change the shape of the path.
Some of the decoration types can be further customized with parameters (see the documentation):
%
\begin{VerbatimOut}{\jobname.tmp}
\centering
% \usetikzlibrary{decorations.pathmorphing}

\begin{tikzpicture}
\draw[decorate,decoration=zigzag] (0,0)--++(3,0);
\draw[decorate,decoration={coil,segment length=0.38cm}]
  (4,0) to[bend right] ++(3,1);
\end{tikzpicture}
\end{VerbatimOut}
\ShowExampleBelow[2]

Second, the decoration can replace the path with symbols.
The easiest way is the following.
Note how the markings rotate with the path:
%
\begin{VerbatimOut}{\jobname.tmp}
\centering
% \usetikzlibrary{decorations.shapes}

\begin{tikzpicture}
\draw[decorate,decoration=crosses] (0,0) to[bend right] ++(3,1);
\draw[decorate,decoration=triangles] (4,0) to[bend right] ++(3,1);
\end{tikzpicture}
\end{VerbatimOut}
\ShowExampleBelow[2]
%
This replaces the whole path.
To add markings on top of a path, you can either draw the path and markings separately,
or use the more powerful marking extension.

The super-generic marking library already has quite a lot of syntax,
which can be referenced from \cite[Section~50.6]{tikz}.
The important part is that the \verb|postaction={decorate}|
argument preserves the path instead of replacing it.
%
\begin{VerbatimOut}{\jobname.tmp}
\centering
% \usetikzlibrary{decorations.markings}

\begin{tikzpicture}
\draw[postaction={decorate}, decoration={markings,
	mark=between positions 0.1 and 1.0 step 4mm with {\arrow{stealth}}
	}] (4,0) to[bend right] ++(3,1);
\end{tikzpicture}
\end{VerbatimOut}
\ShowExampleBelow[2]




%
\subsection{Commutative diagrams}\label{sec:tikz-cd}

In certain fields of mathematics, everything is explainable with sufficiently many arrows.
If you are one of those people, then you will love the \pkg{tikz-cd} package.
It can also be loaded through \verb|\usetikzlibrary{cd}|.
%
\begin{VerbatimOut}{\jobname.tmp}
\centering
% \usetikzlibrary{babel} % Might be needed, see below
% \usetikzlibrary{cd}

\begin{tikzcd}
A \arrow[r] \arrow[d, dotted] & B \arrow[d, Rightarrow, bend left]\\
C \arrow[r, maps to, "\psi"]  & D
\end{tikzcd}
\end{VerbatimOut}
\ShowExampleBelow[2]

\begin{gotcha}
If you get an error message when you try to add a label (like \verb|"\psi"| above),
you should try loading the \verb|babel| TikZ library.
This relates to the handling of quotation marks by \pkg[TikZ compatibility]{babel} package;
the TikZ support library fixes up the handling inside TikZ environments.
\end{gotcha}

There is a nice online editor for commutative diagrams by Yichuan Shen.%
\footnote{\url{tikzcd.yichuanshen.de/}}
It only supports a subset of \pkg{tikz-cd} features,
but is a nice and visual way to edit the diagram before exporting its \LaTeX{} code.
(It also supports \emph{importing} code for further editing!)
